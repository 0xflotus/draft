%!TEX root = std.tex

\chapter{Bibliography}

\begin{itemize}
\renewcommand{\labelitemi}{---}
\item
  Bjarne Stroustrup,
  \doccite{The \Cpp{} Programming Language, second edition}, Chapter R.
  Addison-Wesley Publishing Company, ISBN 0-201-53992-6, copyright \copyright 1991 AT\&T
\item
  Brian W. Kernighan and Dennis M. Ritchie,
  \doccite{The C Programming Language}, Appendix A.
  Prentice-Hall, 1978, ISBN 0-13-110163-3, copyright \copyright 1978 AT\&T
\item
  P.J. Plauger,
  \doccite{The Draft Standard \Cpp{} Library}.
  Prentice-Hall, ISBN 0-13-117003-1, copyright \copyright 1995 P.J. Plauger)
\item
  IANA Time Zone Database,
  available at \url{https://www.iana.org/time-zones}
\item
  ISO/IEC 10967-1:2012,
  \doccite{Information technology --- Language independent arithmetic ---
    Part 1: Integer and floating point arithmetic}
\item
  ISO 4217:2015,
  \doccite{Codes for the representation of currencies}
\end{itemize}

The arithmetic specification described in ISO/IEC 10967-1:2012 is
called \defn{LIA-1} in this document.

% FIXME: For unknown reasons, hanging paragraphs are not indented within our
% glossaries by default.
\let\realglossitem\glossitem
\renewcommand{\glossitem}[4]{\hangpara{4em}{1}\realglossitem{#1}{#2}{#3}{#4}}

\clearpage
\renewcommand{\glossaryname}{Cross references}
\renewcommand{\preglossaryhook}{This annex lists each clause or subclause label and the
corresponding clause or subclause number and page number, in alphabetical order by label.\\}
\twocolglossary
\renewcommand{\leftmark}{\glossaryname}
{
\raggedright
\printglossary[xrefindex]
}

\clearpage
%!TEX root = std.tex

\newcommand{\secref}[1]{\hyperref[\indexescape{#1}]{\indexescape{#1}}}

% Turn off page numbers for this glossary, they're not useful.
\newcommand{\swallow}[1]{}
\changeglossnumformat[xrefdelta]{|swallow}

\newcommand{\oldxref}[2]{\glossary[xrefdelta]{\indexescape{#1}}{#2}}
\newcommand{\removedxref}[1]{\oldxref{#1}{\textit{removed}}}
\newcommand{\movedxrefs}[2]{\oldxref{#1}{\textit{see} #2}}
\newcommand{\movedxref}[2]{\movedxrefs{#1}{\secref{#2}}}
\newcommand{\movedxrefii}[3]{\movedxrefs{#1}{\secref{#2}, \secref{#3}}}
\newcommand{\movedxrefiii}[4]{\movedxrefs{#1}{\secref{#2}, \secref{#3}, \secref{#4}}}
\newcommand{\deprxref}[1]{\oldxref{#1}{\textit{see} \secref{depr.#1}}}

% Removed features.
%\removedxref{removed.label}
\removedxref{variant.traits}

% [facets.examples] was removed.
\removedxref{facets.examples}

% Renamed sections.
%\movedxref{old.label}{new.label}
%\movedxrefii{old.label}{new.label.1}{new.label.2}
%\movedxrefiii{old.label}{new.label.1}{new.label.2}{new.label.3}
%\movedxrefs{old.label}{new place (eg \tref{blah})}

% P0588 replaced function prototype scope with function parameter scope.
\movedxref{basic.scope.proto}{basic.scope.param}

\movedxref{expr.pseudo}{expr.prim.id.dtor}

\movedxref{utility.from.chars}{charconv.from.chars}
\movedxref{utility.to.chars}{charconv.to.chars}

% [fs.definitions] and its contents were integrated into the main text.
% Note that ISO C++17 does not contain the [fs.def.*] subclauses.
\movedxrefs{fs.definitions}{%
  \secref{fs.class.path},
  \secref{fs.conform.os},
  \secref{fs.general},
  \secref{fs.path.fmt.cvt},
  \secref{fs.path.generic},
  \secref{fs.race.behavior}}

% Single-item array subclauses were dissolved.
\movedxref{array.size}{array.members}
\movedxref{array.data}{array.members}
\movedxref{array.fill}{array.members}
\movedxref{array.swap}{array.members}

% Contents of [util.smartptr] was integrated into the parent.
\removedxref{util.smartptr}

% Avoid duplication with synopsis.
\movedxref{re.regex.const}{re.regex}

% Single-item [insert.iterators] subclauses were dissolved.
\movedxref{back.insert.iter.cons}{back.insert.iter.ops}
\movedxref{back.insert.iter.op=}{back.insert.iter.ops}
\movedxref{back.insert.iter.op*}{back.insert.iter.ops}
\movedxref{back.insert.iter.op++}{back.insert.iter.ops}

\movedxref{front.insert.iter.cons}{front.insert.iter.ops}
\movedxref{front.insert.iter.op=}{front.insert.iter.ops}
\movedxref{front.insert.iter.op*}{front.insert.iter.ops}
\movedxref{front.insert.iter.op++}{front.insert.iter.ops}

\movedxref{insert.iter.cons}{insert.iter.ops}
\movedxref{insert.iter.op=}{insert.iter.ops}
\movedxref{insert.iter.op*}{insert.iter.ops}
\movedxref{insert.iter.op++}{insert.iter.ops}

% Single-item [reverse.iterators] subclauses were dissolved.
\movedxref{reverse.iter.op=}{reverse.iter.cons}

\movedxref{reverse.iter.op==}{reverse.iter.cmp}
\movedxref{reverse.iter.op<}{reverse.iter.cmp}
\movedxref{reverse.iter.op!=}{reverse.iter.cmp}
\movedxref{reverse.iter.op>}{reverse.iter.cmp}
\movedxref{reverse.iter.op>=}{reverse.iter.cmp}
\movedxref{reverse.iter.op<=}{reverse.iter.cmp}

\movedxref{reverse.iter.op.star}{reverse.iter.elem}
\movedxref{reverse.iter.opref}{reverse.iter.elem}
\movedxref{reverse.iter.opindex}{reverse.iter.elem}

\movedxref{reverse.iter.op+}{reverse.iter.nav}
\movedxref{reverse.iter.op-}{reverse.iter.nav}
\movedxref{reverse.iter.op++}{reverse.iter.nav}
\movedxref{reverse.iter.op+=}{reverse.iter.nav}
\movedxref{reverse.iter.op\dcr}{reverse.iter.nav}
\movedxref{reverse.iter.op-=}{reverse.iter.nav}

\movedxref{reverse.iter.opdiff}{reverse.iter.nonmember}
\movedxref{reverse.iter.opsum}{reverse.iter.nonmember}
\movedxref{reverse.iter.make}{reverse.iter.nonmember}

\removedxref{reverse.iter.ops}

% Single-item [move.iterators] subclauses were dissolved.
\movedxref{move.iter.op=}{move.iter.cons}
\movedxref{move.iter.op.const}{move.iter.cons}

\movedxref{move.iter.op.star}{move.iter.elem}
\movedxref{move.iter.op.ref}{move.iter.elem}
\movedxref{move.iter.op.index}{move.iter.elem}

\movedxref{move.iter.op.+}{move.iter.nav}
\movedxref{move.iter.op.-}{move.iter.nav}
\movedxref{move.iter.op.incr}{move.iter.nav}
\movedxref{move.iter.op.+=}{move.iter.nav}
\movedxref{move.iter.op.decr}{move.iter.nav}
\movedxref{move.iter.op.-=}{move.iter.nav}

\removedxref{move.iter.ops}

% Individual swap sections were removed.
\removedxref{deque.special}
\removedxref{forwardlist.spec}
\removedxref{list.special}
\removedxref{vector.special}
\removedxref{map.special}
\removedxref{multimap.special}
\removedxref{set.special}
\removedxref{multiset.special}
\removedxref{unord.map.swap}
\removedxref{unord.multimap.swap}
\removedxref{unord.set.swap}
\removedxref{unord.multiset.swap}
\movedxref{re.regex.nmswap}{re.regex.nonmemb}

% Deprecated features were removed.
\removedxref{depr.except.spec}
\removedxref{depr.cpp.headers}
\removedxref{depr.uncaught}
\removedxref{depr.func.adaptor.binding}
\removedxref{depr.weak.result_type}
\removedxref{depr.func.adaptor.typedefs}
\removedxref{depr.negators}
\removedxref{depr.default.allocator}
\removedxref{depr.storage.iterator}
\removedxref{depr.temporary.buffer}
\removedxref{depr.util.smartptr.shared.obs}

% Deprecated <cfoo> headers were removed for some <foo.h> headers
\movedxref{depr.ccomplex.syn}{depr.complex.h.syn}
\movedxref{depr.cstdalign.syn}{depr.stdalign.h.syn}
\movedxref{depr.cstdbool.syn}{depr.stdbool.h.syn}
\movedxref{depr.ctgmath.syn}{depr.tgmath.h.syn}

\movedxref{class.copy}{class.mem}

% Top-level clause merging caused some Annex A subclauses to vanish.
\movedxref{gram.decl}{gram.dcl.decl}
\movedxref{gram.derived}{gram.class}
\movedxref{gram.special}{gram.class}

% Top-level clause merging caused some Annex C subclauses to vanish, too.
\movedxref{diff.conv}{diff.expr}
\movedxref{diff.decl}{diff.dcl}
\movedxref{diff.special}{diff.class}
\movedxref{diff.cpp03.conv}{diff.cpp03.expr}
\movedxref{diff.cpp03.dcl.decl}{diff.cpp03.dcl.dcl}
\movedxref{diff.cpp03.special}{diff.cpp03.class}
\movedxref{diff.cpp11.dcl.decl}{diff.cpp11.dcl.dcl}
\movedxref{diff.cpp14.decl}{diff.cpp14.dcl.dcl}
\movedxref{diff.cpp14.special}{diff.cpp14.class}

% P1148R0 consolidated some Clause 20 subclauses.
\movedxref{string.rfind}{string.find}
\movedxref{string.find.first.of}{string.find}
\movedxref{string.find.last.of}{string.find}
\movedxref{string.find.first.not.of}{string.find}
\movedxref{string.find.last.not.of}{string.find}
\movedxref{string.op+=}{string.op.append}
\movedxref{string.op+}{string.op.plus}
\movedxref{string.operator==}{string.cmp}
\movedxref{string.op!=}{string.cmp}
\movedxref{string.op<}{string.cmp}
\movedxref{string.op>}{string.cmp}
\movedxref{string.op<=}{string.cmp}
\movedxref{string.op>=}{string.cmp}

\movedxref{istream::sentry}{istream.sentry}
\movedxref{ostream::sentry}{ostream.sentry}
\movedxref{ios::failure}{ios.failure}
\movedxref{ios::fmtflags}{ios.fmtflags}
\movedxref{ios::iostate}{ios.iostate}
\movedxref{ios::openmode}{ios.openmode}
\movedxref{ios::seekdir}{ios.seekdir}
\movedxref{ios::Init}{ios.init}

\movedxref{thread.decaycopy}{expos.only.func}

\movedxref{iterator.container}{iterator.range}

% Remove underscores in stable labels.
\movedxref{alg.all_of}{alg.all.of}
\movedxref{alg.any_of}{alg.any.of}
\movedxref{alg.is_permutation}{alg.is.permutation}
\movedxref{alg.none_of}{alg.none.of}
\movedxref{any.bad_any_cast}{any.bad.any.cast}
\movedxref{char.traits.specializations.char16_t}{char.traits.specializations.char16.t}
\movedxref{char.traits.specializations.char32_t}{char.traits.specializations.char32.t}
\movedxref{comparisons.equal_to}{comparisons.equal.to}
\movedxref{comparisons.greater_equal}{comparisons.greater.equal}
\movedxref{comparisons.less_equal}{comparisons.less.equal}
\movedxref{comparisons.not_equal_to}{comparisons.not.equal.to}
\movedxref{condition_variable.syn}{condition.variable.syn}
\movedxref{depr.static_constexpr}{depr.static.constexpr}
\movedxref{forward_list.syn}{forward.list.syn}
\movedxref{fs.class.directory_entry}{fs.class.directory.entry}
\movedxref{fs.class.directory_iterator}{fs.class.directory.iterator}
\movedxref{fs.class.file_status}{fs.class.file.status}
\movedxref{fs.class.filesystem_error}{fs.class.filesystem.error}
\movedxref{fs.enum.file_type}{fs.enum.file.type}
\movedxref{fs.file_status.cons}{fs.file.status.cons}
\movedxref{fs.file_status.mods}{fs.file.status.mods}
\movedxref{fs.file_status.obs}{fs.file.status.obs}
\movedxref{fs.filesystem_error.members}{fs.filesystem.error.members}
\movedxref{fs.op.copy_file}{fs.op.copy.file}
\movedxref{fs.op.copy_symlink}{fs.op.copy.symlink}
\movedxref{fs.op.create_directories}{fs.op.create.directories}
\movedxref{fs.op.create_directory}{fs.op.create.directory}
\movedxref{fs.op.create_dir_symlk}{fs.op.create.dir.symlk}
\movedxref{fs.op.create_hard_lk}{fs.op.create.hard.lk}
\movedxref{fs.op.create_symlink}{fs.op.create.symlink}
\movedxref{fs.op.current_path}{fs.op.current.path}
\movedxref{fs.op.file_size}{fs.op.file.size}
\movedxref{fs.op.hard_lk_ct}{fs.op.hard.lk.ct}
\movedxref{fs.op.is_block_file}{fs.op.is.block.file}
\movedxref{fs.op.is_char_file}{fs.op.is.char.file}
\movedxref{fs.op.is_directory}{fs.op.is.directory}
\movedxref{fs.op.is_empty}{fs.op.is.empty}
\movedxref{fs.op.is_fifo}{fs.op.is.fifo}
\movedxref{fs.op.is_other}{fs.op.is.other}
\movedxref{fs.op.is_regular_file}{fs.op.is.regular.file}
\movedxref{fs.op.is_socket}{fs.op.is.socket}
\movedxref{fs.op.is_symlink}{fs.op.is.symlink}
\movedxref{fs.op.last_write_time}{fs.op.last.write.time}
\movedxref{fs.op.read_symlink}{fs.op.read.symlink}
\movedxref{fs.op.remove_all}{fs.op.remove.all}
\movedxref{fs.op.resize_file}{fs.op.resize.file}
\movedxref{fs.op.status_known}{fs.op.status.known}
\movedxref{fs.op.symlink_status}{fs.op.symlink.status}
\movedxref{fs.op.temp_dir_path}{fs.op.temp.dir.path}
\movedxref{fs.op.weakly_canonical}{fs.op.weakly.canonical}
\movedxref{func.not_fn}{func.not.fn}
\movedxref{futures.future_error}{futures.future.error}
\movedxref{futures.shared_future}{futures.shared.future}
\movedxref{futures.unique_future}{futures.unique.future}
\movedxref{initializer_list.syn}{initializer.list.syn}
\movedxref{optional.comp_with_t}{optional.comp.with.t}
\movedxref{sf.cmath.assoc_laguerre}{sf.cmath.assoc.laguerre}
\movedxref{sf.cmath.assoc_legendre}{sf.cmath.assoc.legendre}
\movedxref{sf.cmath.comp_ellint_1}{sf.cmath.comp.ellint.1}
\movedxref{sf.cmath.comp_ellint_2}{sf.cmath.comp.ellint.2}
\movedxref{sf.cmath.comp_ellint_3}{sf.cmath.comp.ellint.3}
\movedxref{sf.cmath.cyl_bessel_i}{sf.cmath.cyl.bessel.i}
\movedxref{sf.cmath.cyl_bessel_j}{sf.cmath.cyl.bessel.j}
\movedxref{sf.cmath.cyl_bessel_k}{sf.cmath.cyl.bessel.k}
\movedxref{sf.cmath.cyl_neumann}{sf.cmath.cyl.neumann}
\movedxref{sf.cmath.ellint_1}{sf.cmath.ellint.1}
\movedxref{sf.cmath.ellint_2}{sf.cmath.ellint.2}
\movedxref{sf.cmath.ellint_3}{sf.cmath.ellint.3}
\movedxref{sf.cmath.riemann_zeta}{sf.cmath.riemann.zeta}
\movedxref{sf.cmath.sph_bessel}{sf.cmath.sph.bessel}
\movedxref{sf.cmath.sph_legendre}{sf.cmath.sph.legendre}
\movedxref{sf.cmath.sph_neumann}{sf.cmath.sph.neumann}
\movedxref{shared_mutex.syn}{shared.mutex.syn}
\movedxref{system_error.syn}{system.error.syn}
\movedxref{time.traits.duration_values}{time.traits.duration.values}
\movedxref{time.traits.is_fp}{time.traits.is.fp}
\movedxref{utility.as_const}{utility.as.const}

% Dissolved subclause.
\movedxref{func.wrap.badcall.const}{func.wrap.badcall}

% Shortened label
\movedxref{language.support}{support}

% Dissolved subclause
\movedxref{intro.ack}{intro.refs}

% Deprecated features.
\deprxref{util.smartptr.shared.atomic}
\deprxref{res.on.required}
\deprxref{fs.path.factory}
\movedxref{operators}{depr.relops}

% Collapsed subclauses with no siblings.
\movedxref{depr.iterator.primitives}{depr.iterator}
\movedxref{depr.iterator.basic}{depr.iterator}
\movedxref{unreachable.sentinels}{unreachable.sentinel}
\movedxref{conversions}{conversions.character}

\renewcommand{\glossaryname}{Cross references from ISO \CppXVII{}}
\renewcommand{\preglossaryhook}{All clause and subclause labels from
ISO \CppXVII{} (ISO/IEC 14882:2017, \doccite{Programming Languages --- \Cpp{}})
are present in this document, with the exceptions described below.\\}
\renewcommand{\leftmark}{\glossaryname}
{
\raggedright
\printglossary[xrefdelta]
}

\clearpage
\renewcommand{\leftmark}{\indexname}
{
\raggedright
\printindex[generalindex]
}

\clearpage
\renewcommand{\indexname}{Index of grammar productions}
\renewcommand{\preindexhook}{The first bold page number for each entry is the page in the
general text where the grammar production is defined. The second bold page number is the
corresponding page in the Grammar summary\iref{gram}. Other page numbers refer to pages where the grammar production is mentioned in the general text.\\}
\renewcommand{\leftmark}{\indexname}
{
\raggedright
\printindex[grammarindex]
}

\clearpage
\renewcommand{\preindexhook}{The bold page number for each entry refers to
the page where the synopsis of the header is shown.\\}
\renewcommand{\indexname}{Index of library headers}
\renewcommand{\leftmark}{\indexname}
{
\raggedright
\printindex[headerindex]
}

\clearpage
\renewcommand{\preindexhook}{}
\renewcommand{\indexname}{Index of library names}
\renewcommand{\leftmark}{\indexname}
{
\raggedright
\printindex[libraryindex]
}

\clearpage
\renewcommand{\preindexhook}{The bold page number for each entry is the page
where the concept is defined.
Other page numbers refer to pages where the concept is mentioned in the general text.\\}
\renewcommand{\indexname}{Index of library concepts}
\renewcommand{\leftmark}{\indexname}
{
\raggedright
\printindex[conceptindex]
}

\clearpage
\renewcommand{\preindexhook}{The entries in this index are rough descriptions; exact
specifications are at the indicated page in the general text.\\}
\renewcommand{\indexname}{Index of implementation-defined behavior}
\renewcommand{\leftmark}{Index of impl.-def. behavior}
{
\raggedright
\printindex[impldefindex]
}
