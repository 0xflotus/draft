%!TEX root = std.tex
\rSec0[library]{Library introduction}

\rSec1[library.general]{General}

\pnum
This Clause describes the contents of the
\defnx{\Cpp{} standard library}{library!\Cpp{} standard},
how a well-formed \Cpp{} program makes use of the library, and
how a conforming implementation may provide the entities in the library.

\pnum
The following subclauses describe the method of
description\iref{description} and organization\iref{organization} of the
library. \ref{requirements}, \ref{\firstlibchapter}
through \ref{\lastlibchapter}, and \ref{depr} specify the contents of the
library, as well as library requirements and constraints on both well-formed
\Cpp{} programs and conforming implementations.

\pnum
Detailed specifications for each of the components in the library are in
\ref{\firstlibchapter}--\ref{\lastlibchapter}, as shown in
\tref{library.categories}.

\begin{floattable}{Library categories}{library.categories}
{ll}
\topline
\hdstyle{Clause}        & \hdstyle{Category}          \\ \capsep
\ref{support}           & Language support library    \\
\ref{concepts}          & Concepts library            \\
\ref{diagnostics}       & Diagnostics library         \\
\ref{utilities}         & General utilities library   \\
\ref{strings}           & Strings library             \\
\ref{containers}        & Containers library          \\
\ref{iterators}         & Iterators library           \\
\ref{ranges}            & Ranges library              \\
\ref{algorithms}        & Algorithms library          \\
\ref{numerics}          & Numerics library            \\
\ref{time}              & Time library                \\
\ref{localization}      & Localization library        \\
\ref{input.output}      & Input/output library        \\
\ref{re}                & Regular expressions library \\
\ref{atomics}           & Atomic operations library   \\
\ref{thread}            & Thread support library      \\
\end{floattable}

\pnum
The language support library\iref{support} provides components that are
required by certain parts of the \Cpp{} language, such as memory allocation~(\ref{expr.new},
\ref{expr.delete}) and exception processing\iref{except}.

\pnum
The concepts library\iref{concepts} describes library components that \Cpp{}
programs may use to perform compile-time validation of template arguments and
perform function dispatch based on properties of types.

\pnum
The diagnostics library\iref{diagnostics} provides a consistent framework for
reporting errors in a \Cpp{} program, including predefined exception classes.

\pnum
The general utilities library\iref{utilities} includes components used
by other library elements, such as a predefined storage allocator for dynamic
storage management\iref{basic.stc.dynamic}, and components used
as infrastructure
in \Cpp{} programs,
such as tuples, function wrappers, and time facilities.

\pnum
The strings library\iref{strings} provides support for manipulating text represented
as sequences of type \tcode{char},
sequences of type \tcode{char8_t},
sequences of type \tcode{char16_t},
sequences of type \tcode{char32_t},
sequences of type \tcode{wchar_t},
and sequences of any other character-like type.

\pnum
The localization library\iref{localization} provides extended internationalization
support for text processing.

\pnum
The containers\iref{containers}, iterators\iref{iterators}, ranges\iref{ranges},
and algorithms\iref{algorithms} libraries provide a \Cpp{} program with access
to a subset of the most widely used algorithms and data structures.

\pnum
The numerics library\iref{numerics} provides
numeric algorithms and complex number components that extend support for numeric processing.
The
\tcode{valarray}
component provides support for
\textit{n}-at-a-time
processing,
potentially implemented as parallel operations on platforms that support such processing.
The random number component provides facilities for generating pseudo-random numbers.

\pnum
The input/output library\iref{input.output} provides the
\tcode{iostream}
components that are the primary mechanism for \Cpp{} program input and output.
They can be used with other elements of the library, particularly
strings, locales, and iterators.

\pnum
The regular expressions library\iref{re} provides regular expression matching and searching.

\pnum
The atomic operations library\iref{atomics} allows more fine-grained
concurrent access to shared data than is possible with locks.

\pnum
The thread support library\iref{thread} provides components to create
and manage threads, including mutual exclusion and interthread communication.

\rSec1[library.c]{The C standard library}

\pnum
The \Cpp{} standard library also makes available the facilities of the C standard library,
\indextext{library!C standard}%
suitably adjusted to ensure static type safety.

\pnum
The descriptions of many library functions rely on the C standard library for
the semantics of those functions.
In some cases,
the signatures specified in this document
may be different from the signatures in the C standard library,
and additional overloads may be declared in this document,
but the behavior and the preconditions
(including any preconditions implied by the use of an
ISO C \tcode{restrict} qualifier)
are the same unless otherwise stated.

\rSec1[description]{Method of description}

\rSec2[description.general]{General}

\pnum
Subclause \ref{description} describes the conventions used to specify the \Cpp{} standard
library. \ref{structure} describes the structure of
\ref{\firstlibchapter} through \ref{\lastlibchapter} and
\ref{depr}. \ref{conventions} describes other editorial conventions.

\rSec2[structure]{Structure of each clause}

\rSec3[structure.elements]{Elements}

\pnum
Each library clause contains the following elements, as applicable:\footnote{To
save space, items that do not apply to a Clause are omitted.
For example, if a Clause does not specify any requirements,
there will be no ``Requirements'' subclause.}

\begin{itemize}
\item Summary
\item Requirements
\item Detailed specifications
\item References to the C standard library
\end{itemize}

\rSec3[structure.summary]{Summary}

\pnum
The Summary provides a synopsis of the category, and introduces the first-level subclauses.
Each subclause also provides a summary, listing the headers specified in the
subclause and the library entities provided in each header.

\pnum
The contents of the summary and the detailed specifications include:

\begin{itemize}
\item macros
\item values
\item types and alias templates
\item classes and class templates
\item functions and function templates
\item objects and variable templates
\item concepts
\end{itemize}

\rSec3[structure.requirements]{Requirements}

\pnum
\indextext{requirements}%
Requirements describe constraints that shall be met by a \Cpp{} program that extends the standard library.
Such extensions are generally one of the following:

\begin{itemize}
\item Template arguments
\item Derived classes
\item Containers, iterators, and algorithms that meet an interface convention or
  model a concept
\end{itemize}

\pnum
The string and iostream components use an explicit representation of operations
required of template arguments. They use a class template \tcode{char_traits} to
define these constraints.

\pnum
Interface convention requirements are stated as generally as possible. Instead
of stating ``class \tcode{X} has to define a member function \tcode{operator++()}'', the
interface requires ``for any object \tcode{x} of class \tcode{X}, \tcode{++x} is
defined''. That is, whether the operator is a member is unspecified.

\pnum
Requirements are stated in terms of well-defined expressions that define valid terms of
the types that meet the requirements. For every set of well-defined expression
requirements there is either a named concept or a table that specifies an initial set of the valid expressions and
their semantics. Any generic algorithm\iref{algorithms} that uses the
well-defined expression requirements is described in terms of the valid expressions for
its template type parameters.

\pnum
The library specification uses a typographical convention for naming
requirements. Names in \textit{italic} type that begin with the prefix
\oldconcept{} refer to sets of well-defined expression requirements typically
presented in tabular form, possibly with additional prose semantic requirements.
For example, \oldconcept{Destructible}~(\tref{cpp17.destructible}) is such a named
requirement. Names in \tcode{constant width} type refer to library concepts
which are presented as a concept definition\iref{temp}, possibly with additional
prose semantic requirements. For example,
\libconcept{destructible}\iref{concept.destructible}
is such a named requirement.

\pnum
Template argument requirements are sometimes referenced by name.
See~\ref{type.descriptions}.

\pnum
In some cases the semantic requirements are presented as \Cpp{} code.
Such code is intended as a
specification of equivalence of a construct to another construct, not
necessarily as the way the construct
must be implemented.\footnote{Although in some cases the code given is
unambiguously the optimum implementation.}

\pnum
Required operations of any concept defined in this document need not be
total functions; that is, some arguments to a required operation may
result in the required semantics failing to be met.
\begin{example}
The required \tcode{<} operator of the \libconcept{totally_ordered}
concept\iref{concept.totallyordered} does not meet the
semantic requirements of that concept when operating on NaNs.
\end{example}
This does not affect whether a type models the concept.

\pnum
A declaration may explicitly impose requirements through its associated
constraints\iref{temp.constr.decl}. When the associated constraints refer to a
concept\iref{temp.concept}, the semantic constraints specified for that concept
are additionally imposed on the use of the declaration.

\rSec3[structure.specifications]{Detailed specifications}

\pnum
The detailed specifications each contain the following elements:%

\begin{itemize}
\item name and brief description
\item synopsis (class definition or function declaration, as appropriate)
\item restrictions on template arguments, if any
\item description of class invariants
\item description of function semantics
\end{itemize}

\pnum
Descriptions of class member functions follow the order (as
appropriate):\footnote{To save space, items that do not apply to a class are omitted.
For example, if a class does not specify any comparison operator functions, there
will be no ``Comparison operator functions'' subclause.}

\begin{itemize}
\item constructor(s) and destructor
\item copying, moving \& assignment functions
\item comparison operator functions
\item modifier functions
\item observer functions
\item operators and other non-member functions
\end{itemize}

\pnum
Descriptions of function semantics contain the following elements (as
appropriate):\footnote{To save space, elements that do not apply to a function are omitted.
For example, if a function specifies no
preconditions, there will be no \expects element.}

\begin{itemize}
\item
\constraints
the conditions for the function's participation
in overload resolution\iref{over.match}.
\begin{note}
Failure to meet such a condition results in the function's silent non-viability.
\end{note}
\begin{example}
An implementation might express such a condition
via a \grammarterm{constraint-expression}\iref{temp.constr.decl}.
\end{example}

\item
\mandates
the conditions that, if not met, render the program ill-formed.
\begin{example}
An implementation might express such a condition
via the \grammarterm{constant-expression}
in a \grammarterm{static_assert-declaration}\iref{dcl.pre}.
If the diagnostic is to be emitted only after the function
has been selected by overload resolution,
an implementation might express such a condition
via a \grammarterm{constraint-expression}\iref{temp.constr.decl}
and also define the function as deleted.
\end{example}

\item
\expects
the conditions
that the function assumes to hold whenever it is called;
violation of any preconditions results in undefined behavior.

\item
\effects
the actions performed by the function.

\item
\sync
the synchronization operations\iref{intro.multithread} applicable to the function.

\item
\ensures
the conditions (sometimes termed observable results)
established by the function.

\item
\returns
a description of the value(s) returned by the function.

\item
\throws
any exceptions thrown by the function, and the conditions that would cause the exception.

\item
\complexity
the time and/or space complexity of the function.

\item
\remarks
additional semantic constraints on the function.

\item
\errors
the error conditions for error codes reported by the function.
\end{itemize}

\pnum
Whenever the \Fundescx{Effects} element specifies that the semantics of some function
\tcode{F} are \term{Equivalent to} some code sequence, then the various elements are
interpreted as follows.
If \tcode{F}'s semantics specifies any \Fundescx{Constraints} or \Fundescx{Mandates} elements,
then those requirements are logically imposed prior to the \term{equivalent-to} semantics.
Next, the semantics of the code sequence are determined by the
\Fundescx{Constraints}, \Fundescx{Mandates}, \Fundescx{Preconditions}, \Fundescx{Effects},
\Fundescx{Synchronization}, \Fundescx{Postconditions}, \Fundescx{Returns}, \Fundescx{Throws},
\Fundescx{Complexity}, \Fundescx{Remarks}, and \Fundescx{Error conditions}
specified for the function invocations contained in the code sequence.
The value returned from \tcode{F} is specified by \tcode{F}'s \Fundescx{Returns} element,
or if \tcode{F} has no \Fundescx{Returns} element,
a non-\tcode{void} return from \tcode{F} is specified by the
\tcode{return} statements\iref{stmt.return} in the code sequence.
If \tcode{F}'s semantics contains a \Fundescx{Throws},
\Fundescx{Postconditions}, or \Fundescx{Complexity} element,
then that supersedes any occurrences of that element in the code sequence.

\pnum
For non-reserved replacement and handler functions,
\ref{support} specifies two behaviors for the functions in question:
their required and default behavior.
The \defnx{default behavior}{behavior!default}
describes a function definition provided by the implementation.
The \defnx{required behavior}{behavior!required}
describes the semantics of a function definition provided by
either the implementation or a \Cpp{} program.
Where no distinction is explicitly made in the description, the
behavior described is the required behavior.

\pnum
If the formulation of a complexity requirement calls for a negative number of
operations, the actual requirement is zero operations.\footnote{This simplifies
the presentation of complexity requirements in some cases.}

\pnum
Complexity requirements specified in the library clauses are upper bounds,
and implementations that provide better complexity guarantees meet
the requirements.

\pnum
Error conditions specify conditions where a function may fail. The conditions
are listed, together with a suitable explanation, as the \tcode{enum class errc}
constants\iref{syserr}.

\rSec3[structure.see.also]{C library}

\pnum
Paragraphs labeled ``\textsc{See also}'' contain cross-references to the relevant portions
of other standards\iref{intro.refs}.

\rSec2[conventions]{Other conventions}

\rSec3[conventions.general]{General}
\indextext{conventions}%

\pnum
Subclause \ref{conventions} describes several editorial conventions used to describe the contents
of the \Cpp{} standard library.
These conventions are for describing
implementation-defined types\iref{type.descriptions},
and member functions\iref{functions.within.classes}.

\rSec3[expos.only.func]{Exposition-only functions}

\pnum
Several function templates
defined in \ref{\firstlibchapter} through \ref{\lastlibchapter} and \ref{depr}
are only defined for the purpose of exposition.
The declaration of such a function is followed by a comment ending in \expos.

\pnum
The following are defined for exposition only
to aid in the specification of the library:
\indexlibrary{decay-copy@\tcode{\placeholder{decay-copy}}}%
\begin{codeblock}
template<class T> constexpr decay_t<T> @\placeholdernc{decay-copy}@(T&& v)
    noexcept(is_nothrow_convertible_v<T, decay_t<T>>)           // \expos
  { return std::forward<T>(v); }

constexpr auto @\placeholdernc{synth-three-way}@ =
  []<class T, class U>(const T& t, const U& u)
    requires requires {
      { t < u } -> @\exposconcept{boolean-testable}@;
      { u < t } -> @\exposconcept{boolean-testable}@;
    }
  {
    if constexpr (three_way_comparable_with<T, U>) {
      return t <=> u;
    } else {
      if (t < u) return weak_ordering::less;
      if (u < t) return weak_ordering::greater;
      return weak_ordering::equivalent;
    }
  };

template<class T, class U=T>
using @\placeholdernc{synth-three-way-result}@ = decltype(@\placeholdernc{synth-three-way}@(declval<T&>(), declval<U&>()));
\end{codeblock}

\rSec3[type.descriptions]{Type descriptions}

\rSec4[type.descriptions.general]{General}

\pnum
The Requirements subclauses may describe names that are used to specify
constraints on template arguments.\footnote{Examples
from~\ref{utility.requirements} include:
\oldconcept{EqualityComparable},
\oldconcept{LessThanComparable},
\oldconcept{CopyConstructible}.
Examples from~\ref{iterator.requirements} include:
\oldconcept{InputIterator},
\oldconcept{ForwardIterator}.}
These names are used in library Clauses
to describe the types that
may be supplied as arguments by a \Cpp{} program when instantiating template components from
the library.

\pnum
Certain types defined in \ref{input.output} are used to describe implementation-defined types.
\indextext{types!implementation-defined}%
They are based on other types, but with added constraints.

\rSec4[expos.only.types]{Exposition-only types}

\pnum
Several types defined in \ref{\firstlibchapter} through \ref{\lastlibchapter}
and \ref{depr} are defined for the purpose of exposition.
The declaration of such a type is followed by a comment ending in \expos.
\begin{example}
\begin{codeblock}
namespace std {
  extern "C" using @\placeholdernc{some-handler}@ = int(int, void*, double);  // \expos
}
\end{codeblock}
The type placeholder \tcode{\placeholder{some-handler}} can now be used to specify a function
that takes a callback parameter with C language linkage.
\end{example}

\rSec4[enumerated.types]{Enumerated types}

\pnum
Several types defined in \ref{input.output} are
\defnx{enumerated types}{type!enumerated}.
Each enumerated type may be implemented as an enumeration or as a synonym for
an enumeration.\footnote{Such as an integer type, with constant integer
values\iref{basic.fundamental}.}

\pnum
The enumerated type \tcode{\placeholder{enumerated}} can be written:

\begin{codeblock}
enum @\placeholder{enumerated}@ { @$\tcode{\placeholder{V}}_{0}$@, @$\tcode{\placeholder{V}}_{1}$@, @$\tcode{\placeholder{V}}_{2}$@, @$\tcode{\placeholder{V}}_{3}$@, @$\ldots$@ };

inline const @$\tcode{\placeholder{enumerated C}}_{0}$@(@$\tcode{\placeholder{V}}_{0}$@);
inline const @$\tcode{\placeholder{enumerated C}}_{1}$@(@$\tcode{\placeholder{V}}_{1}$@);
inline const @$\tcode{\placeholder{enumerated C}}_{2}$@(@$\tcode{\placeholder{V}}_{2}$@);
inline const @$\tcode{\placeholder{enumerated C}}_{3}$@(@$\tcode{\placeholder{V}}_{3}$@);
  @\vdots@
\end{codeblock}

\pnum
Here, the names $\tcode{\placeholder{C}}_0$,
$\tcode{\placeholder{C}}_1$, etc.\ represent
\defnx{enumerated elements}{enumerated element}
for this particular enumerated type.
\indextext{type!enumerated}%
All such elements have distinct values.

\rSec4[bitmask.types]{Bitmask types}

\pnum
Several types defined in \ref{\firstlibchapter} through \ref{\lastlibchapter}
and \ref{depr} are
\defnx{bitmask types}{type!bitmask}.
Each bitmask type can be implemented as an
enumerated type that overloads certain operators, as an integer type,
or as a
\tcode{bitset}\iref{template.bitset}.
\indextext{type!enumerated}%

\pnum
The bitmask type \tcode{\placeholder{bitmask}} can be written:

\begin{codeblock}
// For exposition only.
// \tcode{int_type} is an integral type capable of representing all values of the bitmask type.
enum @\placeholder{bitmask}@ : int_type {
  @$\tcode{\placeholder{V}}_{0}$@ = 1 << 0, @$\tcode{\placeholder{V}}_{1}$@ = 1 << 1, @$\tcode{\placeholder{V}}_{2}$@ = 1 << 2, @$\tcode{\placeholder{V}}_{3}$@ = 1 << 3, @$\ldots$@
};

inline constexpr @$\tcode{\placeholder{bitmask C}}_{0}$@(@$\tcode{\placeholder{V}}_{0}{}$@);
inline constexpr @$\tcode{\placeholder{bitmask C}}_{1}$@(@$\tcode{\placeholder{V}}_{1}{}$@);
inline constexpr @$\tcode{\placeholder{bitmask C}}_{2}$@(@$\tcode{\placeholder{V}}_{2}{}$@);
inline constexpr @$\tcode{\placeholder{bitmask C}}_{3}$@(@$\tcode{\placeholder{V}}_{3}{}$@);
  @\vdots@

constexpr @\placeholder{bitmask}{}@ operator&(@\placeholder{bitmask}{}@ X, @\placeholder{bitmask}{}@ Y) {
  return static_cast<@\placeholder{bitmask}{}@>(
    static_cast<int_type>(X) & static_cast<int_type>(Y));
}
constexpr @\placeholder{bitmask}{}@ operator|(@\placeholder{bitmask}{}@ X, @\placeholder{bitmask}{}@ Y) {
  return static_cast<@\placeholder{bitmask}{}@>(
    static_cast<int_type>(X) | static_cast<int_type>(Y));
}
constexpr @\placeholder{bitmask}{}@ operator^(@\placeholder{bitmask}{}@ X, @\placeholder{bitmask}{}@ Y){
  return static_cast<@\placeholder{bitmask}{}@>(
    static_cast<int_type>(X) ^ static_cast<int_type>(Y));
}
constexpr @\placeholder{bitmask}{}@ operator~(@\placeholder{bitmask}{}@ X){
  return static_cast<@\placeholder{bitmask}{}@>(~static_cast<int_type>(X));
}
@\placeholder{bitmask}{}@& operator&=(@\placeholder{bitmask}{}@& X, @\placeholder{bitmask}{}@ Y){
  X = X & Y; return X;
}
@\placeholder{bitmask}{}@& operator|=(@\placeholder{bitmask}{}@& X, @\placeholder{bitmask}{}@ Y) {
  X = X | Y; return X;
}
@\placeholder{bitmask}{}@& operator^=(@\placeholder{bitmask}{}@& X, @\placeholder{bitmask}{}@ Y) {
  X = X ^ Y; return X;
}
\end{codeblock}

\pnum
Here, the names $\tcode{\placeholder{C}}_0$,
$\tcode{\placeholder{C}}_1$, etc.\ represent
\defnx{bitmask elements}{bitmask!element}
for this particular bitmask type.
\indextext{type!bitmask}%
All such elements have distinct, nonzero values such that, for any pair $\tcode{\placeholder{C}}_i$
and $\tcode{\placeholder{C}}_j$ where $i \neq j$, \tcode{$\placeholder{C}_i$ \& $\placeholder{C}_i$} is nonzero and
\tcode{$\placeholder{C}_i$ \& $\placeholder{C}_j$} is zero.
Additionally, the value \tcode{0} is used to represent an \defnx{empty bitmask}{bitmask!empty}, in which no
bitmask elements are set.

\pnum
The following terms apply to objects and values of
bitmask types:
\begin{itemize}
\item
To \defnx{set}{bitmask!value!set}
a value \textit{Y} in an object \textit{X}
is to evaluate the expression \textit{X} \tcode{|=} \textit{Y}.
\item
To \defnx{clear}{bitmask!value!clear}
a value \textit{Y} in an object
\textit{X} is to evaluate the expression \textit{X} \tcode{\&= \~}\textit{Y}.
\item
The value \textit{Y} \defnx{is set}{bitmask!value!is set} in the object
\textit{X} if the expression \textit{X} \tcode{\&} \textit{Y} is nonzero.
\end{itemize}

\rSec4[character.seq]{Character sequences}

\rSec5[character.seq.general]{General}

\pnum
The C standard library makes widespread use
\indextext{library!C standard}%
of characters and character sequences that follow a few uniform conventions:

\begin{itemize}
\item
A \defn{letter} is any of the 26 lowercase or 26
\indextext{lowercase}%
\indextext{uppercase}%
uppercase letters in the basic execution character set.
\item
The
\defnx{decimal-point character}{character!decimal-point}
is the
(single-byte) character used by functions that convert between a (single-byte)
character sequence and a value of one of the floating-point types.
It is used
in the character sequence to denote the beginning of a fractional part.
It is
represented in \ref{\firstlibchapter} through \ref{\lastlibchapter}
and \ref{depr} by a period,
\indextext{period}%
\tcode{'.'},
which is
also its value in the \tcode{"C"}
locale, but may change during program
execution by a call to
\tcode{setlocale(int, const char*)},\footnote{declared in
\libheaderref{clocale}.
\indexlibraryglobal{setlocale}%
}
or by a change to a
\tcode{locale}
object, as described in \ref{locales} and \ref{input.output}.
\item
A
\defn{character sequence}
is an array object\iref{dcl.array} \tcode{\placeholdernc{A}} that
can be declared as
\tcode{\placeholdernc{T\;A}[\placeholder{N}]},
where \tcode{\placeholder{T}} is any of the types
\tcode{char},
\tcode{unsigned char},
or
\tcode{signed char}\iref{basic.fundamental}, optionally qualified by any combination of
\tcode{const}
or
\tcode{volatile}.
The initial elements of the
array have defined contents up to and including an element determined by some
predicate.
A character sequence can be designated by a pointer value
\tcode{\placeholder{S}} that points to its first element.
\end{itemize}

\rSec5[byte.strings]{Byte strings}

\indextext{string!null-terminated byte|see{\ntbs{}}}%
\pnum
A \defnx{null-terminated byte string}{NTBS@\ntbs{}},
or \ntbs{},
is a character sequence whose highest-addressed element
with defined content has the value zero
(the \defnx{terminating null character}{character!terminating null});
no other element in the sequence has the value zero.%
\footnote{Many of the objects manipulated by
function signatures declared in
\libheaderref{cstring} are character sequences or \ntbs{}s.
The size of some of these character sequences is limited by
a length value, maintained separately from the character sequence.}

\pnum
The \defnx{length of an \ntbs{}}{NTBS@\ntbs{}!length}
is the number of elements that
precede the terminating null character.
An \defnx{empty \ntbs{}}{NTBS@\ntbs{}!empty}
has a length of zero.

\pnum
The \defnx{value of an \ntbs{}}{NTBS@\ntbs{}!value}
is the sequence of values of the
elements up to and including the terminating null character.

\pnum
A \defnx{static \ntbs{}}{NTBS@\ntbs{}!static}
is an \ntbs{} with
static storage duration.\footnote{A \grammarterm{string-literal}, such as
\tcode{"abc"},
is a static \ntbs{}.}

\rSec5[multibyte.strings]{Multibyte strings}

\indextext{string!null-terminated multibyte|see{\ntmbs{}}}%
\pnum
A \defnx{null-terminated multibyte string}{NTMBS@\ntmbs{}},
or \ntmbs{},
is an \ntbs{} that constitutes a
sequence of valid multibyte characters, beginning and ending in the initial
shift state.\footnote{An \ntbs{} that contains characters only from the
basic execution character set is also an \ntmbs{}.
Each multibyte character then
consists of a single byte.}

\pnum
A \defnx{static \ntmbs{}}{NTMBS@\ntmbs{}!static}
is an \ntmbs{} with static storage duration.

\rSec4[customization.point.object]{Customization Point Object types}

\pnum
A \term{customization point object} is a function object\iref{function.objects}
with a literal class type that interacts with program-defined types while
enforcing semantic requirements on that interaction.

\pnum
The type of a customization point object, ignoring cv-qualifiers, shall model
\libconcept{semiregular}\iref{concepts.object}.

\pnum
All instances of a specific customization point object type shall
be equal\iref{concepts.equality}.

\pnum
The type \tcode{T} of a customization point object shall model
\tcode{\libconcept{invocable}<const T\&, Args...>}\iref{concept.invocable}
when the types in \tcode{Args...} meet the requirements specified in that
customization point object's definition. When the types of \tcode{Args...} do
not meet the customization point object's requirements, \tcode{T} shall not have
a function call operator that participates in overload resolution.

\pnum
Each customization point object type constrains its return type to model a
particular concept.

\pnum
\begin{note}
Many of the customization point objects in the library evaluate function call
expressions with an unqualified name which results in a call to a
program-defined function found by argument dependent name
lookup\iref{basic.lookup.argdep}. To preclude such an expression resulting in a
call to unconstrained functions with the same name in namespace \tcode{std},
customization point objects specify that lookup for these expressions is
performed in a context that includes deleted overloads matching the signatures
of overloads defined in namespace \tcode{std}. When the deleted overloads are
viable, program-defined overloads need be more specialized\iref{temp.func.order}
or more constrained\iref{temp.constr.order} to be used by a customization point
object.
\end{note}

\rSec3[functions.within.classes]{Functions within classes}

\pnum
For the sake of exposition, \ref{\firstlibchapter} through \ref{\lastlibchapter}
and \ref{depr} do not describe copy/move constructors, assignment
operators, or (non-virtual) destructors with the same apparent
semantics as those that can be generated
by default~(\ref{class.copy.ctor}, \ref{class.copy.assign}, \ref{class.dtor}).
\indextext{constructor!copy}%
\indextext{operator!assignment}%
\indextext{destructor}%
It is unspecified whether
the implementation provides explicit definitions for such member function
signatures, or for virtual destructors that can be generated by default.

\rSec3[objects.within.classes]{Private members}

\pnum
\ref{\firstlibchapter} through \ref{\lastlibchapter} and
\ref{depr} do not specify the representation of classes, and intentionally
omit specification of class members\iref{class.mem}. An implementation may
define static or non-static class members, or both, as needed to implement the
semantics of the member functions specified in \ref{\firstlibchapter}
through \ref{\lastlibchapter} and \ref{depr}.

\pnum
For the sake of exposition,
some subclauses provide representative declarations, and semantic requirements, for
private members of classes that meet the external specifications of the classes.
The declarations for such members are
followed by a comment that ends with \expos, as in:

\begin{codeblock}
streambuf* sb;      // \expos
\end{codeblock}

\pnum
An implementation may use any technique that provides equivalent observable behavior.

\rSec1[requirements]{Library-wide requirements}

\rSec2[requirements.general]{General}

\pnum
Subclause \ref{requirements} specifies requirements that apply to the entire \Cpp{} standard library.
\ref{\firstlibchapter} through \ref{\lastlibchapter} and \ref{depr}
specify the requirements of individual entities within the library.

\pnum
Requirements specified in terms of interactions between threads do not apply to
programs having only a single thread of execution.

\pnum
\ref{organization} describes the library's contents and
organization, \ref{using} describes how well-formed \Cpp{} programs gain access to library
entities,
\ref{utility.requirements} describes constraints on types and functions used with
the \Cpp{} standard library,
\ref{constraints} describes constraints on well-formed \Cpp{} programs, and
\ref{conforming} describes constraints on conforming implementations.

\rSec2[organization]{Library contents and organization}

\rSec3[organization.general]{General}

\pnum
\ref{contents} describes the entities and macros defined in the \Cpp{} standard library.
\ref{headers} lists the standard library headers and some constraints on those headers.
\ref{compliance} lists requirements for a freestanding implementation of the \Cpp{}
standard library.

\rSec3[contents]{Library contents}

\pnum
The \Cpp{} standard library provides definitions
for the entities and macros described in the synopses
of the \Cpp{} standard library headers\iref{headers},
unless otherwise specified.

\pnum
All library entities except
\tcode{operator new}
and
\tcode{operator delete}
are defined within the namespace
\tcode{std}
or namespaces nested within namespace
\tcode{std}.\footnote{The C standard library headers\iref{depr.c.headers} also define
names within the global namespace, while the \Cpp{} headers for C library
facilities\iref{headers} can also define names within the global namespace.}%
\indextext{namespace}
It is unspecified whether names declared in a specific namespace are declared
directly in that namespace or in an inline namespace inside that
namespace.\footnote{This gives implementers freedom to use inline namespaces to
support multiple configurations of the library.}

\pnum
Whenever a name \tcode{x} defined in the standard library is mentioned,
the name \tcode{x} is assumed to be fully qualified as
\tcode{::std::x},
unless explicitly described otherwise. For example, if the \effects element
for library function \tcode{F} is described as calling library function \tcode{G},
the function
\tcode{::std::G}
is meant.

\rSec3[headers]{Headers}

\pnum
Each element of the \Cpp{} standard library is declared or defined (as appropriate) in a
\defn{header}.\footnote{A header is not necessarily a source file, nor are the
sequences delimited by \tcode{<} and \tcode{>} in header names necessarily valid source
file names\iref{cpp.include}.}

\pnum
The \Cpp{} standard library provides the
\defnx{\Cpp{} library headers}{header!\Cpp{} library},
shown in \tref{headers.cpp}.

\begin{multicolfloattable}{\Cpp{} library headers}{headers.cpp}
{llll}
\tcode{<algorithm>} \\
\tcode{<any>} \\
\tcode{<array>} \\
\tcode{<atomic>} \\
\tcode{<barrier>} \\
\tcode{<bit>} \\
\tcode{<bitset>} \\
\tcode{<charconv>} \\
\tcode{<chrono>} \\
\tcode{<codecvt>} \\
\tcode{<compare>} \\
\tcode{<complex>} \\
\tcode{<concepts>} \\
\tcode{<condition_variable>} \\
\tcode{<coroutine>} \\
\tcode{<deque>} \\
\tcode{<exception>} \\
\tcode{<execution>} \\
\tcode{<filesystem>} \\
\tcode{<format>} \\
\columnbreak
\tcode{<forward_list>} \\
\tcode{<fstream>} \\
\tcode{<functional>} \\
\tcode{<future>} \\
\tcode{<initializer_list>} \\
\tcode{<iomanip>} \\
\tcode{<ios>} \\
\tcode{<iosfwd>} \\
\tcode{<iostream>} \\
\tcode{<istream>} \\
\tcode{<iterator>} \\
\tcode{<latch>} \\
\tcode{<limits>} \\
\tcode{<list>} \\
\tcode{<locale>} \\
\tcode{<map>} \\
\tcode{<memory>} \\
\tcode{<memory_resource>} \\
\tcode{<mutex>} \\
\tcode{<new>} \\
\columnbreak
\tcode{<numbers>} \\
\tcode{<numeric>} \\
\tcode{<optional>} \\
\tcode{<ostream>} \\
\tcode{<queue>} \\
\tcode{<random>} \\
\tcode{<ranges>} \\
\tcode{<ratio>} \\
\tcode{<regex>} \\
\tcode{<scoped_allocator>} \\
\tcode{<semaphore>} \\
\tcode{<set>} \\
\tcode{<shared_mutex>} \\
\tcode{<source_location>} \\
\tcode{<span>} \\
\tcode{<sstream>} \\
\tcode{<stack>} \\
\tcode{<stdexcept>} \\
\tcode{<stop_token>} \\
\tcode{<streambuf>} \\
\columnbreak
\tcode{<string>} \\
\tcode{<string_view>} \\
\tcode{<strstream>} \\
\tcode{<syncstream>} \\
\tcode{<system_error>} \\
\tcode{<thread>} \\
\tcode{<tuple>} \\
\tcode{<typeindex>} \\
\tcode{<typeinfo>} \\
\tcode{<type_traits>} \\
\tcode{<unordered_map>} \\
\tcode{<unordered_set>} \\
\tcode{<utility>} \\
\tcode{<valarray>} \\
\tcode{<variant>} \\
\tcode{<vector>} \\
\tcode{<version>} \\
\end{multicolfloattable}

\pnum
The facilities of the C standard library are provided in the
\indextext{library!C standard}%
additional headers shown in \tref{headers.cpp.c}.%
\footnote{It is intentional that there is no \Cpp{} header
for any of these C headers:
\libnoheader{stdatomic.h},
\libnoheader{stdnoreturn.h},
\libnoheader{threads.h}.}

\begin{multicolfloattable}{\Cpp{} headers for C library facilities}{headers.cpp.c}
{lllllll}
\tcode{<cassert>} \\
\tcode{<cctype>} \\
\tcode{<cerrno>} \\
\columnbreak
\tcode{<cfenv>} \\
\tcode{<cfloat>} \\
\tcode{<cinttypes>} \\
\columnbreak
\tcode{<climits>} \\
\tcode{<clocale>} \\
\tcode{<cmath>} \\
\columnbreak
\tcode{<csetjmp>} \\
\tcode{<csignal>} \\
\tcode{<cstdarg>} \\
\columnbreak
\tcode{<cstddef>} \\
\tcode{<cstdint>} \\
\tcode{<cstdio>} \\
\columnbreak
\tcode{<cstdlib>} \\
\tcode{<cstring>} \\
\tcode{<ctime>} \\
\columnbreak
\tcode{<cuchar>} \\
\tcode{<cwchar>} \\
\tcode{<cwctype>} \\
\end{multicolfloattable}

\pnum
The headers listed in \tref{headers.cpp}, or,
for a freestanding implementation,
the subset of such headers that are provided by the implementation,
are collectively known as
the \defnadj{importable}{\Cpp{} library headers}.
\begin{note}
Importable \Cpp{} library headers can be
imported as module units\iref{module.import}.
\end{note}
\begin{example}
\begin{codeblock}
import <vector>;                // imports the \tcode{<vector>} header unit
std::vector<int> vi;            // OK
\end{codeblock}
\end{example}

\pnum
Except as noted in \ref{library} through \ref{\lastlibchapter}
and \ref{depr}, the contents of each header \tcode{c\placeholder{name}} is
the same as that of the corresponding header \tcode{\placeholder{name}.h} as
specified in the C standard library\iref{intro.refs}.
In the \Cpp{} standard library, however, the
declarations (except for names which are defined as macros in C) are within
namespace scope\iref{basic.scope.namespace} of the namespace \tcode{std}.
It is unspecified whether these names (including any overloads added in
\ref{\firstlibchapter} through \ref{\lastlibchapter} and \ref{depr})
are first declared within the global namespace scope
and are then injected into namespace \tcode{std} by explicit
\grammarterm{using-declaration}{s}\iref{namespace.udecl}.

\pnum
Names which are defined as macros in C shall be defined as macros in the \Cpp{}
standard library, even if C grants license for implementation as functions.
\begin{note}
The names defined as macros in C include the following:
\tcode{assert}, \tcode{offsetof}, \tcode{setjmp}, \tcode{va_arg},
\tcode{va_end}, and \tcode{va_start}.
\end{note}

\pnum
Names that are defined as functions in C shall be defined as functions in the
\Cpp{} standard library.\footnote{This disallows the practice, allowed in C, of
providing a masking macro in addition to the function prototype. The only way to
achieve equivalent inline behavior in \Cpp{} is to provide a definition as an
extern inline function.}

\pnum
Identifiers that are keywords or operators in \Cpp{} shall not be defined as
macros in \Cpp{} standard library headers.\footnote{In particular, including the
standard header \libheader{iso646.h} has no effect.}

\pnum
\ref{depr.c.headers}, C standard library headers, describes the effects of using
the \tcode{\placeholder{name}.h} (C header) form in a \Cpp{} program.\footnote{ The
\tcode{".h"} headers dump all their names into the global namespace, whereas the
newer forms keep their names in namespace \tcode{std}. Therefore, the newer
forms are the preferred forms for all uses except for \Cpp{} programs which are
intended to be strictly compatible with C. }

\pnum
Annex K of the C standard describes a large number of functions,
with associated types and macros,
which ``promote safer, more secure programming''
than many of the traditional C library functions.
The names of the functions have a suffix of \tcode{_s};
most of them provide the same service
as the C library function with the unsuffixed name,
but generally take an additional argument
whose value is the size of the result array.
If any \Cpp{} header is included,
it is \impldef{whether functions from Annex K of the C standard library
are declared when \Cpp{} headers are included}
whether any of these names
is declared in the global namespace.
(None of them is declared in namespace \tcode{std}.)

\pnum
\tref{c.annex.k.names} lists the Annex K names
that may be declared in some header.
These names are also subject to the restrictions of~\ref{macro.names}.

\begin{multicolfloattable}{C standard Annex K names}{c.annex.k.names}
{llll}
\tcode{abort_handler_s} \\
\tcode{asctime_s} \\
\tcode{bsearch_s} \\
\tcode{constraint_handler_t} \\
\tcode{ctime_s} \\
\tcode{errno_t} \\
\tcode{fopen_s} \\
\tcode{fprintf_s} \\
\tcode{freopen_s} \\
\tcode{fscanf_s} \\
\tcode{fwprintf_s} \\
\tcode{fwscanf_s} \\
\tcode{getenv_s} \\
\tcode{gets_s} \\
\tcode{gmtime_s} \\
\tcode{ignore_handler_s} \\
\tcode{localtime_s} \\
\tcode{L_tmpnam_s} \\
\tcode{mbsrtowcs_s} \\
\columnbreak
\tcode{mbstowcs_s} \\
\tcode{memcpy_s} \\
\tcode{memmove_s} \\
\tcode{memset_s} \\
\tcode{printf_s} \\
\tcode{qsort_s} \\
\tcode{RSIZE_MAX} \\
\tcode{rsize_t} \\
\tcode{scanf_s} \\
\tcode{set_constraint_handler_s} \\
\tcode{snprintf_s} \\
\tcode{snwprintf_s} \\
\tcode{sprintf_s} \\
\tcode{sscanf_s} \\
\tcode{strcat_s} \\
\tcode{strcpy_s} \\
\tcode{strerrorlen_s} \\
\tcode{strerror_s} \\
\tcode{strlen_s} \\
\columnbreak
\tcode{strncat_s} \\
\tcode{strncpy_s} \\
\tcode{strtok_s} \\
\tcode{swprintf_s} \\
\tcode{swscanf_s} \\
\tcode{tmpfile_s} \\
\tcode{TMP_MAX_S} \\
\tcode{tmpnam_s} \\
\tcode{vfprintf_s} \\
\tcode{vfscanf_s} \\
\tcode{vfwprintf_s} \\
\tcode{vfwscanf_s} \\
\tcode{vprintf_s} \\
\tcode{vscanf_s} \\
\tcode{vsnprintf_s} \\
\tcode{vsnwprintf_s} \\
\tcode{vsprintf_s} \\
\tcode{vsscanf_s} \\
\tcode{vswprintf_s} \\
\columnbreak
\tcode{vswscanf_s} \\
\tcode{vwprintf_s} \\
\tcode{vwscanf_s} \\
\tcode{wcrtomb_s} \\
\tcode{wcscat_s} \\
\tcode{wcscpy_s} \\
\tcode{wcsncat_s} \\
\tcode{wcsncpy_s} \\
\tcode{wcsnlen_s} \\
\tcode{wcsrtombs_s} \\
\tcode{wcstok_s} \\
\tcode{wcstombs_s} \\
\tcode{wctomb_s} \\
\tcode{wmemcpy_s} \\
\tcode{wmemmove_s} \\
\tcode{wprintf_s} \\
\tcode{wscanf_s} \\
\end{multicolfloattable}

\rSec3[compliance]{Freestanding implementations}
\indextext{implementation!freestanding|(}%

\pnum
Two kinds of implementations are defined:
\indextext{implementation!hosted}%
hosted and freestanding\iref{intro.compliance};
the kind of the implementation is
\impldef{whether the implementation is hosted or freestanding}.
For a hosted implementation, this document
describes the set of available headers.

\pnum
A freestanding implementation has an
\impldef{headers for freestanding implementation} set of headers. This set shall
include at least the headers shown in \tref{headers.cpp.fs}.

\begin{libsumtab}{\Cpp{} headers for freestanding implementations}{headers.cpp.fs}
\ref{support.types}      & Types                     & \tcode{<cstddef>}          \\ \rowsep
\ref{support.limits}     & Implementation properties &
  \tcode{<cfloat>}, \tcode{<climits>}, \tcode{<limits>}, \tcode{<version>} \\ \rowsep
\ref{cstdint}            & Integer types             & \tcode{<cstdint>}          \\ \rowsep
\ref{support.start.term} & Start and termination     & \tcode{<cstdlib>}          \\ \rowsep
\ref{support.dynamic}    & Dynamic memory management & \tcode{<new>}              \\ \rowsep
\ref{support.rtti}       & Type identification       & \tcode{<typeinfo>}         \\ \rowsep
\ref{support.srcloc}     & Source location           & \tcode{<source_location>}  \\ \rowsep
\ref{support.exception}  & Exception handling        & \tcode{<exception>}        \\ \rowsep
\ref{support.initlist}   & Initializer lists         & \tcode{<initializer_list>} \\ \rowsep
\ref{cmp}                & Comparisons               & \tcode{<compare>}          \\ \rowsep
\ref{support.coroutine}  & Coroutines support        & \tcode{<coroutine>}        \\ \rowsep
\ref{support.runtime}    & Other runtime support     & \tcode{<cstdarg>}          \\ \rowsep
\ref{concepts}           & Concepts library          & \tcode{<concepts>}         \\ \rowsep
\ref{meta}               & Type traits               & \tcode{<type_traits>}      \\ \rowsep
\ref{bit}                & Bit manipulation          & \tcode{<bit>}              \\ \rowsep
\ref{atomics}            & Atomics                   & \tcode{<atomic>}           \\
\end{libsumtab}

\pnum
The supplied version of the header \libheaderref{cstdlib}
shall declare at least the functions
\indexlibraryglobal{abort}%
\tcode{abort},
\indexlibraryglobal{atexit}%
\tcode{atexit},
\indexlibraryglobal{at_quick_exit}%
\tcode{at_quick_exit},
\indexlibraryglobal{exit}%
\tcode{exit},
and
\indexlibraryglobal{quick_exit}%
\tcode{quick_exit}\iref{support.start.term}.
\indextext{implementation!hosted}%
The supplied version of the header \libheaderrefx{atomic}{atomics.syn}
shall meet the same requirements as for a hosted implementation
except that support for
always lock-free integral atomic types\iref{atomics.lockfree}
is \impldef{support for always lock-free integral atomic types in
freestanding environments}, and
whether or not the type aliases \tcode{atomic_signed_lock_free} and
\tcode{atomic_unsigned_lock_free} are defined\iref{atomics.alias}
is \impldef{type aliases \tcode{atomic_signed_lock_free} and
\tcode{atomic_unsigned_lock_free} in freestanding environments}.
The other headers listed in this table
shall meet the same requirements as for a hosted implementation.
\indextext{implementation!freestanding|)}%

\rSec2[using]{Using the library}

\rSec3[using.overview]{Overview}

\pnum
Subclause \ref{using} describes how a \Cpp{} program gains access to the facilities of the
\Cpp{} standard library. \ref{using.headers} describes effects during translation
phase 4, while~\ref{using.linkage} describes effects during phase
8\iref{lex.phases}.

\rSec3[using.headers]{Headers}

\pnum
The entities in the \Cpp{} standard library are defined in headers,
whose contents are made available to a translation unit when it contains the appropriate
\indextext{unit!translation}%
\indextext{\idxcode{\#include}}%
\tcode{\#include}
preprocessing directive\iref{cpp.include}
or the appropriate
\indextext{\idxcode{import}}%
\tcode{import} declaration\iref{module.import}.
\indextext{source file}

\pnum
A translation unit may include library headers in any order\iref{lex.separate}.
\indextext{unit!translation}%
Each may be included more than once, with no effect different from
being included exactly once, except that the effect of including either
\libheaderref{cassert} or \libheaderrefx{assert.h}{depr.c.headers}
depends each time on the lexically current definition of
\indextext{\idxcode{NDEBUG}}%
\indexlibraryglobal{NDEBUG}%
\tcode{NDEBUG}.\footnote{This is the same as the C standard library.}

\pnum
A translation unit shall include a header only outside of any
\indextext{unit!translation}%
declaration or definition and,
in the case of a module unit,
only in its \grammarterm{global-module-fragment}, and
shall include the header or import the corresponding header unit lexically
before the first reference in that translation unit to any of the entities
declared in that header. No diagnostic is required.

\rSec3[using.linkage]{Linkage}

\pnum
Entities in the \Cpp{} standard library have external linkage\iref{basic.link}.
Unless otherwise specified, objects and functions have the default
\tcode{extern "C++"}
linkage\iref{dcl.link}.

\pnum
\indextext{library!C standard}%
Whether a name from the C standard library declared with
external linkage has
\indextext{linkage!external}%
\indextext{header!C library}%
\indextext{\idxcode{extern ""C""}}%
\tcode{extern "C"}
or
\indextext{\idxcode{extern ""C++""}}%
\tcode{extern "C++"}
linkage is \impldef{linkage of names from C standard library}. It is recommended that an
implementation use
\tcode{extern "C++"}
linkage for this purpose.\footnote{The only reliable way to declare an object or
function signature from the C standard library is by including the header that
declares it, notwithstanding the latitude granted in 7.1.4 of the C
Standard.}

\pnum
Objects and functions
defined in the library and required by a \Cpp{} program are included in
the program prior to program startup.

\indextext{startup!program}%
\pnum
See also
replacement functions\iref{replacement.functions},
runtime changes\iref{handler.functions}.

\rSec2[utility.requirements]{Requirements on types and expressions}

\rSec3[utility.requirements.general]{General}

\pnum
\ref{utility.arg.requirements}
describes requirements on types and expressions used to instantiate templates
defined in the \Cpp{} standard library.
\ref{swappable.requirements} describes the requirements on swappable types and
swappable expressions.
\ref{nullablepointer.requirements} describes the requirements on pointer-like
types that support null values.
\ref{hash.requirements} describes the requirements on hash function objects.
\ref{allocator.requirements} describes the requirements on storage
allocators.

\rSec3[utility.arg.requirements]{Template argument requirements}

\pnum
The template definitions in the \Cpp{} standard library
refer to various named requirements whose details are set out in
Tables~\ref{tab:cpp17.equalitycomparable}--\ref{tab:cpp17.destructible}.
In these tables, \tcode{T} is an object or reference type to be
supplied by a \Cpp{} program instantiating a template;
\tcode{a},
\tcode{b}, and
\tcode{c} are values of type (possibly \tcode{const}) \tcode{T};
\tcode{s} and \tcode{t} are modifiable lvalues of type \tcode{T};
\tcode{u} denotes an identifier;
\tcode{rv} is an rvalue of type \tcode{T};
and \tcode{v} is an lvalue of type (possibly \tcode{const}) \tcode{T} or an rvalue of
type \tcode{const T}.

\pnum
In general, a default constructor is not required. Certain container class
member function signatures specify \tcode{T()} as a default argument.
\tcode{T()} shall be a well-defined expression\iref{dcl.init} if one of those
signatures is called using the default argument\iref{dcl.fct.default}.

\begin{oldconcepttable}{EqualityComparable}{}{cpp17.equalitycomparable}
{x{1in}x{1in}p{3in}}
\topline
\hdstyle{Expression}  &   \hdstyle{Return type} &   \rhdr{Requirement} \\ \capsep
\tcode{a == b}  &
convertible to \tcode{bool} &
\tcode{==} is an equivalence relation,
that is, it has the following properties:
\begin{itemize}
\item
For all \tcode{a}, \tcode{a == a}.
\item
If \tcode{a == b}, then \tcode{b == a}.
\item
If \tcode{a == b} and \tcode{b == c}, then \tcode{a == c}.
\end{itemize} \\
\end{oldconcepttable}

\begin{oldconcepttable}{LessThanComparable}{}{cpp17.lessthancomparable}
{x{1in}x{1in}p{3in}}
\topline
\hdstyle{Expression}  &   \hdstyle{Return type} &   \hdstyle{Requirement} \\ \capsep
\tcode{a < b}   &
convertible to \tcode{bool} &
\tcode{<} is a strict weak ordering relation\iref{alg.sorting}    \\
\end{oldconcepttable}

\enlargethispage{-3\baselineskip}
\begin{oldconcepttable}{DefaultConstructible}{}{cpp17.defaultconstructible}
{x{2.15in}p{3in}}
\topline
\hdstyle{Expression}        &     \hdstyle{Post-condition}  \\ \capsep
\tcode{T t;}      &     object \tcode{t} is default-initialized   \\ \rowsep
\tcode{T u\{\};}    &     object \tcode{u} is value-initialized or aggregate-initialized \\ \rowsep
\tcode{T()}\br\tcode{T\{\}}  &  an object of type \tcode{T} is value-initialized
                                or aggregate-initialized \\
\end{oldconcepttable}

\begin{oldconcepttable}{MoveConstructible}{}{cpp17.moveconstructible}
{p{1in}p{4.15in}}
\topline
\hdstyle{Expression}          &   \hdstyle{Post-condition}  \\ \capsep
\tcode{T u = rv;}    &   \tcode{u} is equivalent to the value of \tcode{rv} before the construction\\ \rowsep
\tcode{T(rv)}       &
  \tcode{T(rv)} is equivalent to the value of \tcode{rv} before the construction \\ \rowsep
\multicolumn{2}{|p{5.3in}|}{
  \tcode{rv}'s state is unspecified
  \begin{note}
\tcode{rv} must still meet the requirements of the library
  component that is using it. The operations listed in those requirements must
  work as specified whether \tcode{rv} has been moved from or not.
\end{note}
}\\
\end{oldconcepttable}

\begin{oldconcepttable}{CopyConstructible}{ (in addition to \oldconcept{MoveConstructible})}{cpp17.copyconstructible}
{p{1in}p{4.15in}}
\topline
\hdstyle{Expression}          &   \hdstyle{Post-condition}  \\ \capsep
\tcode{T u = v;}     &   the value of \tcode{v} is unchanged and is equivalent to \tcode{ u}\\ \rowsep
\tcode{T(v)}        &
  the value of \tcode{v} is unchanged and is equivalent to \tcode{T(v)} \\
\end{oldconcepttable}

\begin{oldconcepttable}{MoveAssignable}{}{cpp17.moveassignable}
{p{1in}p{1in}p{1in}p{1.9in}}
\topline
\hdstyle{Expression} & \hdstyle{Return type} & \hdstyle{Return value} & \hdstyle{Post-condition} \\ \capsep
\tcode{t = rv}  &   \tcode{T\&} &   \tcode{t}       &
  If \tcode{t} and \tcode{rv} do not refer to the same object,
  \tcode{t} is equivalent to the value of \tcode{rv} before the assignment\\ \rowsep
\multicolumn{4}{|p{5.3in}|}{
  \tcode{rv}'s state is unspecified.
  \begin{note}
  \tcode{rv} must still meet the requirements of the library
  component that is using it, whether or not \tcode{t} and \tcode{rv} refer to the same object.
  The operations listed in those requirements must
  work as specified whether \tcode{rv} has been moved from or not.
\end{note}
}\\
\end{oldconcepttable}

\begin{oldconcepttable}{CopyAssignable}{ (in addition to \oldconcept{MoveAssignable})}{cpp17.copyassignable}
{p{1in}p{1in}p{1in}p{1.9in}}
\topline
\hdstyle{Expression} & \hdstyle{Return type} & \hdstyle{Return value} & \hdstyle{Post-condition} \\ \capsep
\tcode{t = v}   &   \tcode{T\&} &   \tcode{t}   &   \tcode{t} is equivalent to \tcode{v}, the value of \tcode{v} is unchanged\\
\end{oldconcepttable}

\begin{oldconcepttable}{Destructible}{}{cpp17.destructible}
{p{1in}p{4.15in}}
\topline
\hdstyle{Expression}      &   \hdstyle{Post-condition}  \\ \capsep
\tcode{u.\~T()} &   All resources owned by \tcode{u} are reclaimed, no exception is propagated. \\ \rowsep
\multicolumn{2}{|l|}{
  \begin{note}
  Array types and non-object types are not \oldconcept{Destructible}.
  \end{note}
} \\
\end{oldconcepttable}

\rSec3[swappable.requirements]{Swappable requirements}

\pnum
This subclause provides definitions for swappable types and expressions. In these
definitions, let \tcode{t} denote an expression of type \tcode{T}, and let \tcode{u}
denote an expression of type \tcode{U}.

\pnum
An object \tcode{t} is \defn{swappable with} an object \tcode{u} if and only if:
\begin{itemize}
\item the expressions \tcode{swap(t, u)} and \tcode{swap(u, t)} are valid when
evaluated in the context described below, and

\item these expressions have the following effects:

\begin{itemize}
\item the object referred to by \tcode{t} has the value originally held by \tcode{u} and
\item the object referred to by \tcode{u} has the value originally held by \tcode{t}.
\end{itemize}
\end{itemize}

\pnum
The context in which \tcode{swap(t, u)} and \tcode{swap(u, t)} are evaluated shall
ensure that a binary non-member function named ``swap'' is selected via overload
resolution\iref{over.match} on a candidate set that includes:
\begin{itemize}
\item the two \tcode{swap} function templates defined in
\libheaderref{utility} and

\item the lookup set produced by argument-dependent lookup\iref{basic.lookup.argdep}.
\end{itemize}
\begin{note}
If \tcode{T} and \tcode{U} are both fundamental types or arrays of
fundamental types and the declarations from the header \libheader{utility} are in
scope, the overall lookup set described above is equivalent to that of the
qualified name lookup applied to the expression \tcode{std::swap(t, u)} or
\tcode{std::swap(u, t)} as appropriate.
\end{note}
\begin{note}
It is unspecified whether a library component that has a swappable
requirement includes the header \libheader{utility} to ensure an appropriate
evaluation context.
\end{note}

\pnum
An rvalue or lvalue \tcode{t} is \defn{swappable} if and only if \tcode{t} is
swappable with any rvalue or lvalue, respectively, of type \tcode{T}.

\pnum
A type \tcode{X} meeting any of the iterator requirements\iref{iterator.requirements}
meets the \oldconcept{ValueSwappable} requirements if,
for any dereferenceable object
\tcode{x} of type \tcode{X},
\tcode{*x} is swappable.

\pnum
\begin{example}
User code can ensure that the evaluation of \tcode{swap} calls
is performed in an appropriate context under the various conditions as follows:
\begin{codeblock}
#include <utility>

// Requires: \tcode{std::forward<T>(t)} shall be swappable with \tcode{std::forward<U>(u)}.
template<class T, class U>
void value_swap(T&& t, U&& u) {
  using std::swap;
  swap(std::forward<T>(t), std::forward<U>(u)); // OK: uses ``swappable with'' conditions
                                                // for rvalues and lvalues
}

// Requires: lvalues of \tcode{T} shall be swappable.
template<class T>
void lv_swap(T& t1, T& t2) {
  using std::swap;
  swap(t1, t2);                                 // OK: uses swappable conditions for lvalues of type \tcode{T}
}

namespace N {
  struct A { int m; };
  struct Proxy { A* a; };
  Proxy proxy(A& a) { return Proxy{ &a }; }

  void swap(A& x, Proxy p) {
    std::swap(x.m, p.a->m);                     // OK: uses context equivalent to swappable
                                                // conditions for fundamental types
  }
  void swap(Proxy p, A& x) { swap(x, p); }      // satisfy symmetry constraint
}

int main() {
  int i = 1, j = 2;
  lv_swap(i, j);
  assert(i == 2 && j == 1);

  N::A a1 = { 5 }, a2 = { -5 };
  value_swap(a1, proxy(a2));
  assert(a1.m == -5 && a2.m == 5);
}
\end{codeblock}
\end{example}

\rSec3[nullablepointer.requirements]{\oldconcept{NullablePointer} requirements}

\pnum
A \oldconcept{NullablePointer} type is a pointer-like type that supports null values.
A type \tcode{P} meets the \oldconcept{\-Nullable\-Pointer} requirements if:
\begin{itemize}
\item \tcode{P} meets the \oldconcept{EqualityComparable},
\oldconcept{DefaultConstructible}, \oldconcept{CopyConstructible}, \oldconcept{\-Copy\-Assign\-able},
and \oldconcept{Destructible} requirements,

\item lvalues of type \tcode{P} are swappable\iref{swappable.requirements},

\item the expressions shown in \tref{cpp17.nullablepointer} are
valid and have the indicated semantics, and

\item \tcode{P} meets all the other requirements of this subclause.
\end{itemize}

\pnum
A value-initialized object of type \tcode{P} produces the null value of the type.
The null value shall be equivalent only to itself. A default-initialized object
of type \tcode{P} may have an indeterminate value.
\begin{note}
Operations involving
indeterminate values might cause undefined behavior.
\end{note}

\pnum
An object \tcode{p} of type \tcode{P} can be contextually converted to
\tcode{bool}\iref{conv}. The effect shall be as if \tcode{p != nullptr}
had been evaluated in place of \tcode{p}.

\pnum
No operation which is part of the \oldconcept{NullablePointer} requirements shall exit
via an exception.

\pnum
In \tref{cpp17.nullablepointer}, \tcode{u} denotes an identifier, \tcode{t}
denotes a non-\tcode{const} lvalue of type \tcode{P}, \tcode{a} and \tcode{b}
denote values of type (possibly \tcode{const}) \tcode{P}, and \tcode{np} denotes
a value of type (possibly \tcode{const}) \tcode{std::nullptr_t}.

\begin{oldconcepttable}{NullablePointer}{}{cpp17.nullablepointer}
{lll}
\topline
\lhdr{Expression} & \chdr{Return type} & \rhdr{Operational semantics} \\ \capsep
\tcode{P u(np);}\br           &
                              &
  \ensures \tcode{u == nullptr}  \\
\tcode{P u = np;}             &
                              &
                              \\ \rowsep

\tcode{P(np)}                 &
                              &
  \ensures \tcode{P(np) == nullptr}  \\ \rowsep

\tcode{t = np}                &
  \tcode{P\&}                 &
  \ensures \tcode{t == nullptr}  \\ \rowsep

\tcode{a != b}                &
  contextually convertible to \tcode{bool}  &
  \tcode{!(a == b)}           \\ \rowsep

\tcode{a == np}               &
  contextually convertible to \tcode{bool}  &
  \tcode{a == P()}            \\
\tcode{np == a}               &
                              &
                              \\ \rowsep
\tcode{a != np}               &
  contextually convertible to \tcode{bool}  &
  \tcode{!(a == np)}          \\
\tcode{np != a}               &
                              &
                              \\
\end{oldconcepttable}

\rSec3[hash.requirements]{\oldconcept{Hash} requirements}

\pnum
A type \tcode{H} meets the \defnoldconcept{Hash} requirements if:
\begin{itemize}
\item it is a function object type\iref{function.objects},
\item it meets the \oldconcept{CopyConstructible} (\tref{cpp17.copyconstructible}) and
  \oldconcept{Destructible} (\tref{cpp17.destructible}) requirements, and
\item the expressions shown in \tref{cpp17.hash}
are valid and have the indicated semantics.
\end{itemize}

\pnum
Given \tcode{Key} is an argument type for function objects of type \tcode{H}, in
\tref{cpp17.hash} \tcode{h} is a value of type (possibly \tcode{const}) \tcode{H},
\tcode{u} is an lvalue of type \tcode{Key}, and \tcode{k} is a value of a type convertible to
(possibly \tcode{const}) \tcode{Key}.

\begin{oldconcepttable}{Hash}{}{cpp17.hash}
{llp{.55\hsize}}
\topline
\lhdr{Expression} & \chdr{Return type} & \rhdr{Requirement} \\ \capsep
\tcode{h(k)}      &
  \tcode{size_t}  &
  The value returned shall depend only on the argument \tcode{k} for the duration of
  the program.
\begin{note}
Thus all evaluations of the expression \tcode{h(k)} with the
  same value for \tcode{k} yield the same result for a given execution of the program.
  \end{note}
  For two different
  values \tcode{t1} and \tcode{t2}, the probability that \tcode{h(t1)} and \tcode{h(t2)}
  compare equal should be very small, approaching \tcode{1.0 / numeric_limits<size_t>::max()}.
\\ \rowsep
\tcode{h(u)}      &
  \tcode{size_t}  &
  Shall not modify \tcode{u}. \\
\end{oldconcepttable}

\rSec3[allocator.requirements]{\oldconcept{Allocator} requirements}

\rSec4[allocator.requirements.general]{General}

\indextext{\idxoldconcept{Allocator}}%
\pnum
The library describes a standard set of requirements for \term{allocators},
which are class-type objects that encapsulate the information about an allocation model.
This information includes the knowledge of pointer types, the type of their
difference, the type of the size of objects in this allocation model, as well
as the memory allocation and deallocation primitives for it. All of the
string types\iref{strings},
containers\iref{containers} (except \tcode{array}),
string buffers and string streams\iref{input.output}, and
\tcode{match_results}\iref{re} are parameterized in terms of
allocators.

\begin{shortlibreqtab2}
{Descriptive variable definitions}
{allocator.req.var}
\topline
\lhdr{Variable} &   \rhdr{Definition}   \\  \capsep
\tcode{T, U, C}    &   any \cv-unqualified object type\iref{basic.types}       \\ \rowsep
\tcode{X}       &   an allocator class for type \tcode{T}   \\ \rowsep
\tcode{Y}       &   the corresponding allocator class for type \tcode{U}    \\ \rowsep
\tcode{XX}      &   the type \tcode{allocator_traits<X>}    \\ \rowsep
\tcode{YY}      &   the type \tcode{allocator_traits<Y>}    \\ \rowsep
\tcode{a, a1, a2}   &   lvalues of type \tcode{X}      \\ \rowsep
\tcode{u}       &   the name of a variable being declared      \\ \rowsep
\tcode{b}       &   a value of type \tcode{Y}           \\ \rowsep
\tcode{c}       &   a pointer of type \tcode{C*} through which indirection is valid \\ \rowsep
\tcode{p}       &   a value of type \tcode{XX::pointer}, obtained
by calling \tcode{a1.allocate}, where \tcode{a1 == a}   \\ \rowsep
\tcode{q}       &   a value of type \tcode{XX::const_pointer}
obtained by conversion from a value \tcode{p}           \\ \rowsep
\tcode{r}       &   a value of type \tcode{T\&}
obtained by the expression \tcode{*p}                   \\ \rowsep
\tcode{w}       &   a value of type \tcode{XX::void_pointer} obtained by
  conversion from a value \tcode{p}  \\ \rowsep
\tcode{x}       &   a value of type \tcode{XX::const_void_pointer} obtained by
  conversion from a value \tcode{q} or a value \tcode{w}  \\ \rowsep
\tcode{y}       &   a value of type \tcode{XX::const_void_pointer} obtained by
conversion from a result value of \tcode{YY::allocate}, or else a value of
type (possibly \tcode{const}) \tcode{std::nullptr_t}  \\ \rowsep
\tcode{n}       &   a value of type \tcode{XX::size_type}    \\ \rowsep
\tcode{Args}    &   a template parameter pack               \\ \rowsep
\tcode{args}    &   a function parameter pack with the pattern \tcode{Args\&\&} \\
\end{shortlibreqtab2}

\pnum
The class template \tcode{allocator_traits}\iref{allocator.traits} supplies
a uniform interface to all allocator types.
\tref{allocator.req.var} describes the types manipulated
through allocators. \tref{cpp17.allocator}
describes the requirements on allocator types
and thus on types used to instantiate \tcode{allocator_traits}. A requirement
is optional if the last column of
\tref{cpp17.allocator} specifies a default for a
given expression. Within the standard library \tcode{allocator_traits}
template, an optional requirement that is not supplied by an allocator is
replaced by the specified default expression. A user specialization of
\tcode{allocator_traits} may provide different defaults and may provide
defaults for different requirements than the primary template. Within
Tables~\ref{tab:allocator.req.var} and~\ref{tab:cpp17.allocator},
the use of \tcode{move} and \tcode{forward} always refers to \tcode{std::move}
and \tcode{std::forward}, respectively.

\begin{libreqtab4d}
{\oldconcept{Allocator} requirements}
{cpp17.allocator}
\\ \topline
\lhdr{Expression}   &   \chdr{Return type}  &   \chdr{Assertion/note} & \rhdr{Default}       \\
                    &                       &   \chdr{pre-/post-condition}  &   \\ \capsep
\endfirsthead
\continuedcaption\\
\hline
\lhdr{Expression}   &   \chdr{Return type}  &   \chdr{Assertion/note} & \rhdr{Default}       \\
                    &                       &   \chdr{pre-/post-condition}  &   \\ \capsep
\endhead
\tcode{X::pointer}          &          &   & \tcode{T*} \\ \rowsep

\tcode{X::const_pointer}    &
   &
  \tcode{X::pointer} is convertible to \tcode{X::const_pointer}   &
  \tcode{pointer_traits<X::\brk{}pointer>::\brk{}rebind<const T>}             \\ \rowsep

\tcode{X::void_pointer}\br\tcode{Y::void_pointer} &
                                                  &
  \tcode{X::pointer} is convertible to \tcode{X::void_pointer}.
  \tcode{X::void_pointer} and \tcode{Y::void_pointer} are the same type.  &
  \tcode{pointer_traits<X::\brk{}pointer>::\brk{}rebind<void>} \\ \rowsep

\tcode{X::const_void_pointer}\br\tcode{Y::const_void_pointer} &
                                                  &
  \tcode{X::pointer}, \tcode{X::const_pointer}, and \tcode{X::void_pointer} are convertible to \tcode{X::const_void_pointer}.
  \tcode{X::const_void_pointer} and \tcode{Y::const_void_pointer} are the same type.  &
  \tcode{pointer_traits<X::\brk{}pointer>::\brk{}rebind<const void>} \\ \rowsep

\tcode{X::value_type}       &
  Identical to \tcode{T}    &   & \\ \rowsep

\tcode{X::size_type}        &
  unsigned integer type     &
  a type that can represent the size of the largest object in the allocation model &
  \tcode{make_unsigned_t<X::\brk{}difference_type>} \\ \rowsep

\tcode{X::difference_type}  &
  signed integer type       &
  a type that can represent the difference between any two pointers
    in the allocation model &
  \tcode{pointer_traits<X::\brk{}pointer>::\brk{}difference_type} \\ \rowsep

\tcode{typename X::template rebind<U>::other}   &
  \tcode{Y}                 &
  For all \tcode{U} (including \tcode{T}), \tcode{Y::template rebind<T>::other}
    is \tcode{X}.           &
  See Note A, below.        \\ \rowsep

\tcode{*p}                  &
  \tcode{T\&}               && \\ \rowsep

\tcode{*q}                  &
  \tcode{const T\&}         &
  \tcode{*q} refers to the same object as \tcode{*p}. & \\ \rowsep

\tcode{p->m}                &
  type of \tcode{T::m}      &
  \expects \tcode{(*p).m} is well-defined.\br
  equivalent to \tcode{(*p).m}  & \\ \rowsep

\tcode{q->m}                &
  type of \tcode{T::m}      &
  \expects \tcode{(*q).m} is well-defined.\br
  equivalent to \tcode{(*q).m}  & \\ \rowsep

\tcode{static_cast<\brk{}X::pointer\brk{}>(w)}  &
  \tcode{X::pointer}                &
  \tcode{static_cast<X::pointer>(w) == p} & \\ \rowsep

\tcode{static_cast<\brk{}X::const_pointer\brk{}>(x)}  &
  \tcode{X::const_pointer}                &
  \tcode{static_cast<} \tcode{X::const_pointer\brk{}>(x) == q} & \\ \rowsep

\tcode{pointer_traits<\brk{}X::pointer\brk{}>::pointer_to(r)}  &
  \tcode{X::pointer}        &
  same as \tcode{p}         & \\ \rowsep

\tcode{a.allocate(n)}   &   \tcode{X::pointer}  &
Memory is allocated for an array of \tcode{n} \tcode{T}
and such an object is created
but array elements are not constructed.
\begin{example}
When reusing storage denoted by some pointer value \tcode{p},
\tcode{launder(reinterpret_cast<T*>(new (p) byte[n * sizeof(T)]))}
can be used to implicitly create a suitable array object
and obtain a pointer to it.
\end{example}
\tcode{allocate} may throw an appropriate exception.%
\footnote{It is intended that \tcode{a.allocate} be an efficient means
of allocating a single object of type \tcode{T}, even when \tcode{sizeof(T)}
is small. That is, there is no need for a container to maintain its own
free list.}
\begin{note}
If \tcode{n == 0}, the return value is unspecified.
\end{note}
&  \\ \rowsep

\tcode{a.allocate(n, y)}    &
  \tcode{X::pointer}        &
  Same as \tcode{a.allocate(n)}. The use of \tcode{y} is unspecified, but
    it is intended as an aid to locality. &
  \tcode{a.allocate(n)}     \\ \rowsep

\tcode{a.deallocate(p,n)}   &
  (not used)                &
  \expects \tcode{p} is a value returned by an earlier call
  to \tcode{allocate} that has not been invalidated by
  an intervening call to \tcode{deallocate}. \tcode{n}
  matches the value passed to \tcode{allocate} to obtain this memory.\br
  \throws Nothing.          &  \\ \rowsep

\tcode{a.max_size()}        &
  \tcode{X::size_type}      &
  the largest value that can meaningfully be passed to \tcode{X::allocate()}  &
  \tcode{numeric_limits<size_type>::max() / sizeof\brk{}(value_type)}  \\ \rowsep

\tcode{a1 == a2}            &
  \tcode{bool}              &
  Returns \tcode{true} only if storage allocated from each can
    be deallocated via the other. \tcode{operator==} shall be reflexive, symmetric,
    and transitive, and shall not exit via an exception. &  \\ \rowsep

\tcode{a1 != a2}            &
  \tcode{bool}              &
  same as \tcode{!(a1 == a2)}     & \\ \rowsep

\tcode{a == b}              &
  \tcode{bool}              &
  same as \tcode{a ==} \tcode{Y::rebind<T>::other(b)} & \\ \rowsep

\tcode{a != b}              &
  \tcode{bool}              &
  same as \tcode{!(a == b)} & \\ \rowsep

\tcode{X u(a)};            \br
\tcode{X u = a;}           &
                            &
  Shall not exit via an exception.\br
  \ensures \tcode{u == a}     & \\ \rowsep

\tcode{X u(b);}             &
                            &
  Shall not exit via an exception.\br
  \ensures \tcode{Y(u) == b}, \tcode{u == X(b)} &  \\ \rowsep

\tcode{X u(std::move(a));}  \br
\tcode{X u = std::move(a);} &
                            &
  Shall not exit via an exception.\br
  \ensures The value of \tcode{a} is unchanged and is equal to \tcode{u}. & \\ \rowsep

\tcode{X u(std::move(b));}  &
                            &
  Shall not exit via an exception.\br
  \ensures \tcode{u} is equal to the prior value of \tcode{X(b)}. & \\ \rowsep

\tcode{a.construct(c, args)}&
  (not used)                &
  \effects Constructs an object of type \tcode{C} at
    \tcode{c}.              &
  \tcode{construct_at(c,~std::\brk{}forward<Args>\brk{}(args)...)}  \\ \rowsep

\tcode{a.destroy(c)}        &
  (not used)                &
  \effects Destroys the object at \tcode{c}  &
  \tcode{destroy_at(c)}     \\  \rowsep

\tcode{a.select_on_container_copy_construction()} &
  \tcode{X}                 &
  Typically returns either \tcode{a} or \tcode{X()}. &
  \tcode{return a;}         \\ \rowsep

\tcode{X::propagate_on_container_copy_assignment} &
  Identical to or derived from \tcode{true_type} or \tcode{false_type}  &
  \tcode{true_type} only if an allocator of type \tcode{X} should be copied
    when the client container is copy-assigned.
    See Note B, below.   &
  \tcode{false_type}        \\ \rowsep

\tcode{X::propagate_on_container_move_assignment} &
  Identical to or derived from \tcode{true_type} or \tcode{false_type}  &
  \tcode{true_type} only if an allocator of type \tcode{X} should be moved
    when the client container is move-assigned.
    See Note B, below.   &
  \tcode{false_type}        \\ \rowsep

\tcode{X::propagate_on_-} \tcode{container_swap} &
  Identical to or derived from \tcode{true_type} or \tcode{false_type}  &
  \tcode{true_type} only if an allocator of type \tcode{X} should be swapped
    when the client container is swapped.
    See Note B, below.   &
  \tcode{false_type}        \\ \rowsep

\tcode{X::is_always_equal} &
  Identical to or derived from \tcode{true_type} or \tcode{false_type}  &
  \tcode{true_type} only if the expression \tcode{a1 == a2} is guaranteed
    to be \tcode{true} for any two (possibly \tcode{const}) values
    \tcode{a1}, \tcode{a2} of type \tcode{X}.   &
  \tcode{is_empty<X>::\brk{}type}       \\

\end{libreqtab4d}

\pnum
Note A: The member class template \tcode{rebind} in the table above is
effectively a typedef template.
\begin{note}
In general, if
the name \tcode{Allocator} is bound to \tcode{SomeAllocator<T>}, then
\tcode{Allocator::rebind<U>::other} is the same type as
\tcode{SomeAllocator<U>}, where
\tcode{SomeAllocator<T>::value_type} is \tcode{T} and
\tcode{SomeAllocator<U>::\brk{}value_type} is \tcode{U}.
\end{note}
If
\tcode{Allocator} is a class template instantiation of the form
\tcode{SomeAllocator<T, Args>}, where \tcode{Args} is zero or more type
arguments, and \tcode{Allocator} does not supply a \tcode{rebind} member
template, the standard \tcode{allocator_traits} template uses
\tcode{SomeAllocator<U, Args>} in place of \tcode{Allocator::\brk{}rebind<U>::other}
by default. For allocator types that are not template instantiations of the
above form, no default is provided.

\pnum
Note B:
If \tcode{X::propagate_on_container_copy_assignment::value} is \tcode{true},
\tcode{X} shall meet the
\oldconcept{\-Copy\-Assign\-able} requirements (\tref{cpp17.copyassignable})
and the copy operation shall not throw exceptions.
If \tcode{X::propagate_on_container_move_assignment::value} is \tcode{true},
\tcode{X} shall meet the
\oldconcept{\-Move\-Assign\-able} requirements (\tref{cpp17.moveassignable})
and the move operation shall not throw exceptions.
If \tcode{X::propagate_on_container_swap::value} is \tcode{true},
lvalues of type \tcode{X} shall be swappable\iref{swappable.requirements}
and the \tcode{swap} operation shall not throw exceptions.

\pnum
An allocator type \tcode{X} shall meet the
\oldconcept{CopyConstructible} requirements (\tref{cpp17.copyconstructible}).
The \tcode{X::pointer}, \tcode{X::const_pointer}, \tcode{X::void_pointer}, and
\tcode{X::const_void_pointer} types shall meet the
\oldconcept{Nullable\-Pointer} requirements (\tref{cpp17.nullablepointer}).
No constructor,
comparison operator function, copy operation, move operation, or swap operation on
these pointer types shall exit via an exception. \tcode{X::pointer} and \tcode{X::const_pointer} shall also
meet the requirements for
a \oldconcept{RandomAccessIterator}\iref{random.access.iterators} and
the additional requirement that, when \tcode{a} and \tcode{(a + n)} are
dereferenceable pointer values for some integral value \tcode{n},
\begin{codeblock}
addressof(*(a + n)) == addressof(*a) + n
\end{codeblock}
is \tcode{true}.

\pnum
Let \tcode{x1} and \tcode{x2} denote objects of (possibly different) types
\tcode{X::void_pointer}, \tcode{X::const_void_pointer}, \tcode{X::pointer},
or \tcode{X::const_pointer}. Then, \tcode{x1} and \tcode{x2} are
\defn{equivalently-valued} pointer values, if and only if both \tcode{x1} and \tcode{x2}
can be explicitly converted to the two corresponding objects \tcode{px1} and \tcode{px2}
of type \tcode{X::const_pointer}, using a sequence of \tcode{static_cast}s
using only these four types, and the expression \tcode{px1 == px2}
evaluates to \tcode{true}.

\pnum
Let \tcode{w1} and \tcode{w2} denote objects of type \tcode{X::void_pointer}.
Then for the expressions
\begin{codeblock}
w1 == w2
w1 != w2
\end{codeblock}
either or both objects may be replaced by an equivalently-valued object of type
\tcode{X::const_void_pointer} with no change in semantics.

\pnum
Let \tcode{p1} and \tcode{p2} denote objects of type \tcode{X::pointer}.
Then for the expressions
\begin{codeblock}
p1 == p2
p1 != p2
p1 < p2
p1 <= p2
p1 >= p2
p1 > p2
p1 - p2
\end{codeblock}
either or both objects may be replaced by an equivalently-valued object of type
\tcode{X::const_pointer} with no change in semantics.

\pnum
An allocator may constrain the types on which it can be instantiated and the
arguments for which its \tcode{construct} or \tcode{destroy} members may be
called. If a type cannot be used with a particular allocator, the allocator
class or the call to \tcode{construct} or \tcode{destroy} may fail to instantiate.

\pnum
If the alignment associated with a specific over-aligned type is not
supported by an allocator, instantiation of the allocator for that type may
fail. The allocator also may silently ignore the requested alignment.
\begin{note}
Additionally, the member function \tcode{allocate}
for that type can fail by throwing an object of type
\tcode{bad_alloc}.
\end{note}

\pnum
\begin{example}
The following is an allocator class template supporting the minimal
interface that meets the requirements of
\tref{cpp17.allocator}:

\begin{codeblock}
template<class Tp>
struct SimpleAllocator {
  typedef Tp value_type;
  SimpleAllocator(@\textit{ctor args}@);

  template<class T> SimpleAllocator(const SimpleAllocator<T>& other);

  [[nodiscard]] Tp* allocate(std::size_t n);
  void deallocate(Tp* p, std::size_t n);
};

template<class T, class U>
bool operator==(const SimpleAllocator<T>&, const SimpleAllocator<U>&);
template<class T, class U>
bool operator!=(const SimpleAllocator<T>&, const SimpleAllocator<U>&);
\end{codeblock}
\end{example}

\rSec4[allocator.requirements.completeness]{Allocator completeness requirements}

\pnum
If \tcode{X} is an allocator class for type \tcode{T},
\tcode{X} additionally meets the allocator completeness requirements if,
whether or not \tcode{T} is a complete type:
\begin{itemize}
\item \tcode{X} is a complete type, and
\item all the member types of \tcode{allocator_traits<X>}\iref{allocator.traits}
  other than \tcode{value_type} are complete types.
\end{itemize}

\rSec2[constraints]{Constraints on programs}

\rSec3[constraints.overview]{Overview}

\pnum
Subclause \ref{constraints} describes restrictions on \Cpp{} programs that use the facilities of
the \Cpp{} standard library. The following subclauses specify constraints on the
program's use of namespaces\iref{namespace.std}, its use of various reserved
names\iref{reserved.names}, its use of headers\iref{alt.headers}, its use of
standard library classes as base classes\iref{derived.classes}, its
definitions of replacement functions\iref{replacement.functions}, and its
installation of handler functions during execution\iref{handler.functions}.

\rSec3[namespace.constraints]{Namespace use}

\rSec4[namespace.std]{Namespace \tcode{std}}

\pnum
Unless otherwise specified,
the behavior of a \Cpp{} program is undefined if it adds declarations or definitions to namespace
\tcode{std}
or to a namespace within namespace
\tcode{std}.

\pnum
Unless explicitly prohibited,
a program may add a template specialization for
any standard library class template
to namespace
\tcode{std} provided that
(a) the added declaration
depends on at least one program-defined type
and
(b) the specialization meets the standard library requirements
for the original template.\footnote{Any
library code that instantiates other library templates
must be prepared to work adequately with any user-supplied specialization
that meets the minimum requirements of this document.}

\pnum
The behavior of a \Cpp{} program is undefined
if it declares an explicit or partial specialization
of any standard library variable template,
except where explicitly permitted by the specification of that variable template.

\pnum
The behavior of a \Cpp{} program is undefined if it declares
\begin{itemize}
\item an explicit specialization of any member function of a standard
library class template, or

\item an explicit specialization of any member function template of a
standard library class or class template, or

\item an explicit or partial specialization of any member class template
of a standard library class or class template, or

\item a deduction guide for any standard library class template.
\end{itemize}

\pnum
A program may explicitly instantiate
a class template defined in the standard library
only if the declaration
(a) depends on the name of at least one program-defined type
and
(b) the instantiation meets the standard library requirements for the
original template.

\pnum
Let \tcode{\placeholder{F}} denote
a standard library function\iref{global.functions},
a standard library static member function,
or an instantiation
of a standard library function template.
Unless \tcode{\placeholder{F}} is designated
an \defnadj{addressable}{function},
the behavior of a \Cpp{} program is unspecified (possibly ill-formed)
if it explicitly or implicitly attempts
to form a pointer
to \tcode{\placeholder{F}}.
\begin{note}
Possible means of forming such pointers include
application of the unary \tcode{\&} operator\iref{expr.unary.op},
\tcode{addressof}\iref{specialized.addressof},
or
a function-to-pointer standard conversion\iref{conv.func}.
\end{note}
Moreover,
the behavior of a \Cpp{} program is unspecified (possibly ill-formed)
if it attempts to form a reference
to \tcode{\placeholder{F}}
or
if it attempts to form a pointer-to-member designating
either a standard library non-static member function\iref{member.functions}
or an instantiation of a standard library member function template.

\pnum
Other than in namespace \tcode{std}
or in a namespace
within namespace \tcode{std},
a program may provide
an overload for any library function template
designated as a \defn{customization point},
provided that
(a)
the overload's declaration depends
on at least one user-defined type
and
(b)
the overload meets the standard library requirements
for the customization point.%
\footnote{
Any library customization point
must be prepared
to work adequately
with any user-defined overload
that meets the minimum requirements
of this document.
Therefore
an implementation can elect,
under the as-if rule\iref{intro.execution},
to provide any customization point
in the form
of an instantiated function object\iref{function.objects}
even though the customization point's specification
is in the form
of a function template.
The template parameters
of each such function object
and the function parameters
and return type
of the object's \tcode{operator()}
must match those
of the corresponding customization point's specification.}
\begin{note}
This permits
a (qualified or unqualified) call
to the customization point
to invoke the most appropriate overload
for the given arguments.
\end{note}

\pnum
A translation unit shall not declare namespace \tcode{std} to be an inline namespace\iref{namespace.def}.

\rSec4[namespace.posix]{Namespace \tcode{posix}}

\pnum
The behavior of a \Cpp{} program is undefined if it adds declarations or definitions to namespace
\tcode{posix}
or to a namespace within namespace
\tcode{posix}
unless otherwise specified. The namespace \tcode{posix} is reserved for use by
ISO/IEC 9945 and other POSIX standards.

\rSec4[namespace.future]{Namespaces for future standardization}

\pnum
Top-level namespaces whose \grammarterm{namespace-name} consists of \tcode{std}
followed by one or more \grammarterm{digit}{s}\iref{lex.name}
are reserved for future standardization.
The behavior of a \Cpp{} program is undefined if
it adds declarations or definitions to such a namespace.
\begin{example}
The top-level namespace \tcode{std2} is reserved
for use by future revisions of this International Standard.
\end{example}

\rSec3[reserved.names]{Reserved names}%

\rSec4[reserved.names.general]{General}%
\indextext{name!reserved}

\pnum
The \Cpp{} standard library reserves the following kinds of names:
\begin{itemize}
\item macros
\item global names
\item names with external linkage
\end{itemize}

\pnum
If a program declares or defines a name in a context where it is
reserved, other than as explicitly allowed by \ref{library}, its behavior is
undefined.%
\indextext{undefined}

\rSec4[zombie.names]{Zombie names}%
\indextext{name!zombie}%
\indextext{living dead!name of}%
\indextext{brains!names that want to eat your}%

\pnum
In namespace \tcode{std}, the following names are reserved for previous standardization:
\begin{itemize}
\item \indexlibraryzombie{auto_ptr} \tcode{auto_ptr},
\item \indexlibraryzombie{auto_ptr_ref} \tcode{auto_ptr_ref},
\item \indexlibraryzombie{binary_function} \tcode{binary_function},
\item \indexlibraryzombie{binary_negate} \tcode{binary_negate},
\item \indexlibraryzombie{bind1st} \tcode{bind1st},
\item \indexlibraryzombie{bind2nd} \tcode{bind2nd},
\item \indexlibraryzombie{binder1st} \tcode{binder1st},
\item \indexlibraryzombie{binder2nd} \tcode{binder2nd},
\item \indexlibraryzombie{const_mem_fun1_ref_t} \tcode{const_mem_fun1_ref_t},
\item \indexlibraryzombie{const_mem_fun1_t} \tcode{const_mem_fun1_t},
\item \indexlibraryzombie{const_mem_fun_ref_t} \tcode{const_mem_fun_ref_t},
\item \indexlibraryzombie{const_mem_fun_t} \tcode{const_mem_fun_t},
\item \indexlibraryzombie{get_temporary_buffer} \tcode{get_temporary_buffer},
\item \indexlibraryzombie{get_unexpected} \tcode{get_unexpected},
\item \indexlibraryzombie{gets} \tcode{gets},
\item \indexlibraryzombie{is_literal_type} \tcode{is_literal_type},
\item \indexlibraryzombie{is_literal_type_v} \tcode{is_literal_type_v},
\item \indexlibraryzombie{mem_fun1_ref_t} \tcode{mem_fun1_ref_t},
\item \indexlibraryzombie{mem_fun1_t} \tcode{mem_fun1_t},
\item \indexlibraryzombie{mem_fun_ref_t} \tcode{mem_fun_ref_t},
\item \indexlibraryzombie{mem_fun_ref} \tcode{mem_fun_ref},
\item \indexlibraryzombie{mem_fun_t} \tcode{mem_fun_t},
\item \indexlibraryzombie{mem_fun} \tcode{mem_fun},
\item \indexlibraryzombie{not1} \tcode{not1},
\item \indexlibraryzombie{not2} \tcode{not2},
\item \indexlibraryzombie{pointer_to_binary_function} \tcode{pointer_to_binary_function},
\item \indexlibraryzombie{pointer_to_unary_function} \tcode{pointer_to_unary_function},
\item \indexlibraryzombie{ptr_fun} \tcode{ptr_fun},
\item \indexlibraryzombie{random_shuffle} \tcode{random_shuffle},
\item \indexlibraryzombie{raw_storage_iterator} \tcode{raw_storage_iterator},
\item \indexlibraryzombie{result_of} \tcode{result_of},
\item \indexlibraryzombie{result_of_t} \tcode{result_of_t},
\item \indexlibraryzombie{return_temporary_buffer} \tcode{return_temporary_buffer},
\item \indexlibraryzombie{set_unexpected} \tcode{set_unexpected},
\item \indexlibraryzombie{unary_function} \tcode{unary_function},
\item \indexlibraryzombie{unary_negate} \tcode{unary_negate},
\item \indexlibraryzombie{uncaught_exception} \tcode{uncaught_exception},
\item \indexlibraryzombie{unexpected} \tcode{unexpected},
and
\item \indexlibraryzombie{unexpected_handler} \tcode{unexpected_handler}.
\end{itemize}

\pnum
The following names are reserved as member types for previous standardization,
and may not be used as a name for object-like macros in portable code:
\begin{itemize}
\item \indexlibraryzombie{argument_type} \tcode{argument_type},
\item \indexlibraryzombie{first_argument_type} \tcode{first_argument_type},
\item \indexlibraryzombie{io_state} \tcode{io_state},
\item \indexlibraryzombie{open_mode} \tcode{open_mode},
\item \indexlibraryzombie{second_argument_type} \tcode{second_argument_type},
and
\item \indexlibraryzombie{seek_dir} \tcode{seek_dir}.
\end{itemize}

\pnum
The name \indexlibraryzombie{stossc} \tcode{stossc} is reserved as a
member function for previous standardization, and may not be used as a name for
function-like macros in portable code.

\pnum
The header names
\libnoheader{ccomplex},
\libnoheader{ciso646},
\libnoheader{cstdalign},
\libnoheader{cstdbool}, and
\libnoheader{ctgmath}
are reserved for previous standardization.

\rSec4[macro.names]{Macro names}

\pnum
\indextext{\idxcode{\#undef}}%
\indextext{unit!translation}%
A translation unit that includes a standard library header shall not
\tcode{\#define} or \tcode{\#undef} names declared in any standard
library header.

\pnum
\indextext{unit!translation}%
A translation unit shall not \tcode{\#define} or \tcode{\#undef}
names lexically identical
to keywords,
to the identifiers listed in \tref{lex.name.special}, or
to the \grammarterm{attribute-token}{s} described in~\ref{dcl.attr},
except that the names \tcode{likely} and \tcode{unlikely} may be
defined as function-like macros~\iref{cpp.replace}.

\rSec4[extern.names]{External linkage}

\pnum
Each name declared as an object with external linkage
\indextext{linkage!external}%
in a header is reserved to the implementation to designate that library
object with external linkage,%
\indextext{linkage!external}\footnote{The list of such reserved names includes
\tcode{errno}, declared or defined in \libheaderref{cerrno}.}
both in namespace \tcode{std} and in the global namespace.

\pnum
Each
\indextext{function!global}%
global function signature declared with
\indextext{linkage!external}%
external linkage in a header is reserved to the
implementation to designate that function signature with
\indextext{linkage!external}%
external linkage.\footnote{The list of such reserved function
signatures with external linkage includes
\indexlibraryglobal{setjmp}%
\tcode{setjmp(jmp_buf)},
declared or defined in \libheaderref{csetjmp},
and
\indexlibraryglobal{va_end}%
\indexlibraryglobal{va_list}%
\tcode{va_end(va_list)},
declared or defined in
\libheaderref{cstdarg}.}

\pnum
Each name from the C standard library declared with external linkage
\indextext{linkage!external}%
is reserved to the implementation for use as a name with
\indextext{header!C}%
\indextext{\idxcode{extern ""C""}}%
\tcode{extern "C"}
linkage,
both in namespace \tcode{std} and in the global namespace.

\pnum
Each function signature from the C standard library declared with
\indextext{linkage!external}%
external linkage
is reserved to the implementation for use as
a function signature with both
\indextext{\idxcode{extern ""C""}}%
\tcode{extern "C"}
and
\indextext{\idxcode{extern ""C++""}}%
\tcode{extern "C++"}
linkage,\footnote{The function signatures declared in
\indextext{Amendment 1}%
\libheaderref{cuchar},
\libheaderref{cwchar},
and
\libheaderref{cwctype}
are always reserved, notwithstanding the restrictions imposed in subclause
4.5.1 of Amendment 1 to the C Standard for these headers.}
or as a name of namespace scope in the global namespace.

\rSec4[extern.types]{Types}

\pnum
For each type \tcode{T} from the C standard library,
the types
\tcode{::T}
and
\tcode{std::T}
are reserved to the implementation and, when defined,
\tcode{::T}
shall be identical to
\tcode{std::T}.

\rSec4[usrlit.suffix]{User-defined literal suffixes}

\pnum
Literal suffix identifiers\iref{over.literal} that do not start with an underscore are reserved for future standardization.

\rSec3[alt.headers]{Headers}

\pnum
If a file with a name
equivalent to the derived file name for one of the \Cpp{} standard library headers
is not provided as part of the implementation, and a file with that name
is placed in any of the standard places for a source file to be included\iref{cpp.include},
the behavior is undefined.%
\indextext{source file}%
\indextext{undefined}

\rSec3[derived.classes]{Derived classes}

\pnum
Virtual member function signatures defined
\indextext{function!virtual member}%
for a base class in the \Cpp{} standard
\indextext{class!base}%
\indextext{library!\Cpp{} standard}%
library may be overridden in a derived class defined in the program\iref{class.virtual}.

\rSec3[replacement.functions]{Replacement functions}

\pnum
\indextext{definition!alternate}%
\ref{\firstlibchapter} through \ref{\lastlibchapter} and \ref{depr}
describe the behavior of numerous functions defined by
the \Cpp{} standard library.
Under some circumstances,
\indextext{library!\Cpp{} standard}%
however, certain of these function descriptions also apply to replacement functions defined
in the program.

\pnum
A \Cpp{} program may provide the definition for any of the following
dynamic memory allocation function signatures declared in header
\tcode{<new>}~(\ref{basic.stc.dynamic}, \ref{new.syn}):

\indextext{\idxcode{new}!\idxcode{operator}!replaceable}%
\indexlibrarymember{new}{operator}%
\begin{codeblock}
operator new(std::size_t)
operator new(std::size_t, std::align_val_t)
operator new(std::size_t, const std::nothrow_t&)
operator new(std::size_t, std::align_val_t, const std::nothrow_t&)
\end{codeblock}%
\indextext{\idxcode{delete}!\idxcode{operator}!replaceable}%
\indexlibrarymember{delete}{operator}%
\begin{codeblock}
operator delete(void*)
operator delete(void*, std::size_t)
operator delete(void*, std::align_val_t)
operator delete(void*, std::size_t, std::align_val_t)
operator delete(void*, const std::nothrow_t&)
operator delete(void*, std::align_val_t, const std::nothrow_t&)
\end{codeblock}%
\indextext{\idxcode{new}!\idxcode{operator}!replaceable}%
\indexlibrarymember{new}{operator}%
\begin{codeblock}
operator new[](std::size_t)
operator new[](std::size_t, std::align_val_t)
operator new[](std::size_t, const std::nothrow_t&)
operator new[](std::size_t, std::align_val_t, const std::nothrow_t&)
\end{codeblock}%
\indextext{\idxcode{delete}!\idxcode{operator}!replaceable}%
\indexlibrarymember{delete}{operator}%
\begin{codeblock}
operator delete[](void*)
operator delete[](void*, std::size_t)
operator delete[](void*, std::align_val_t)
operator delete[](void*, std::size_t, std::align_val_t)
operator delete[](void*, const std::nothrow_t&)
operator delete[](void*, std::align_val_t, const std::nothrow_t&)
\end{codeblock}

\pnum
The program's definitions are used instead of the default versions supplied by
the implementation\iref{new.delete}.
Such replacement occurs prior to program startup~(\ref{basic.def.odr}, \ref{basic.start}).
\indextext{startup!program}%
The program's declarations shall not be specified as
\tcode{inline}.
No diagnostic is required.

\rSec3[handler.functions]{Handler functions}

\pnum
The \Cpp{} standard library provides a default version of the following handler
function\iref{support}:

\begin{itemize}
\item
\tcode{terminate_handler}
\indexlibraryglobal{terminate_handler}%
\end{itemize}

\pnum
A \Cpp{} program may install different handler functions during execution, by
supplying a pointer to a function defined in the program or the library
as an argument to (respectively):
\begin{itemize}
\item \indexlibraryglobal{set_new_handler}\tcode{set_new_handler}
\item \indexlibraryglobal{set_terminate}\tcode{set_terminate}
\end{itemize}
See also subclauses~\ref{alloc.errors}, Storage allocation errors, and~\ref{support.exception},
Exception handling.

\pnum
A \Cpp{} program can get a pointer to the current handler function by calling the following
functions:

\begin{itemize}
\item
\indexlibraryglobal{get_new_handler}%
\tcode{get_new_handler}
\item
\indexlibraryglobal{get_terminate}
\tcode{get_terminate}
\end{itemize}

\pnum
Calling the \tcode{set_*} and \tcode{get_*} functions shall not incur a data race. A call to
any of the \tcode{set_*} functions shall synchronize with subsequent calls to the same
\tcode{set_*} function and to the corresponding \tcode{get_*} function.

\rSec3[res.on.functions]{Other functions}

\pnum
In certain cases (replacement functions, handler functions, operations on types used to
instantiate standard library template components), the \Cpp{} standard library depends on
components supplied by a \Cpp{} program.
If these components do not meet their requirements, this document places no requirements
on the implementation.

\pnum
In particular, the effects are undefined in the following cases:

\begin{itemize}
\item
For replacement functions\iref{new.delete}, if the installed replacement function does not
implement the semantics of the applicable
\required
paragraph.
\item
For handler functions~(\ref{new.handler}, \ref{terminate.handler}),
if the installed handler function does not implement the semantics of the applicable
\required
paragraph.
\item
For types used as template arguments when instantiating a template component,
if the operations on the type do not implement the semantics of the applicable
\emph{Requirements}
subclause~(\ref{allocator.requirements}, \ref{container.requirements}, \ref{iterator.requirements},
\ref{algorithms.requirements}, \ref{numeric.requirements}).
Operations on such types can report a failure by throwing an exception
unless otherwise specified.
\item
If any replacement function or handler function or destructor operation exits via an exception,
unless specifically allowed
in the applicable
\required
paragraph.
\item
If an incomplete type\iref{basic.types} is used as a template
argument when instantiating a template component or evaluating a concept, unless specifically
allowed for that component.
\end{itemize}

\rSec3[res.on.arguments]{Function arguments}

\pnum
\indextext{restriction}%
\indextext{argument}%
Each of the following applies to all arguments
\indextext{argument}%
to functions defined in the \Cpp{} standard library,%
\indextext{library!\Cpp{} standard}
unless explicitly stated otherwise.

\begin{itemize}
\item
If an argument to a function has an invalid value (such
\indextext{argument}%
as a value outside the domain of the function or a pointer invalid for its
intended use), the behavior is undefined.
\indextext{undefined}%

\item
If a function argument is described as being an array,
\indextext{argument}%
the pointer actually passed to the function shall have a value such that all
address computations and accesses to objects (that would be valid if the
pointer did point to the first element of such an array) are in fact valid.

\item
If a function argument binds to an rvalue reference parameter, the implementation may
assume that this parameter is a unique reference to this argument.
\begin{note}
If the parameter is a generic parameter of the form \tcode{T\&\&} and an lvalue of type
\tcode{A} is bound, the argument binds to an lvalue reference\iref{temp.deduct.call}
and thus is not covered by the previous sentence.
\end{note}
\begin{note}
If a program casts
an lvalue to an xvalue while passing that lvalue to a library function
(e.g., by calling the function with the argument \tcode{std::move(x)}), the program
is effectively asking that function to treat that lvalue as a temporary object.
The implementation
is free to optimize away aliasing checks which might be needed if the argument was
an lvalue.
\end{note}
\end{itemize}

\rSec3[res.on.objects]{Library object access}

\pnum
The behavior of a program is undefined if calls to standard library functions from different
threads may introduce a data race. The conditions under which this may occur are specified
in~\ref{res.on.data.races}.
\begin{note}
Modifying an object of a standard library type that is
shared between threads risks undefined behavior unless objects of that type are explicitly
specified as being shareable without data races or the user supplies a locking mechanism.
\end{note}

\pnum
If an object of a standard library type is accessed, and
the beginning of the object's lifetime\iref{basic.life}
does not happen before the access, or
the access does not happen before the end of the object's lifetime,
the behavior is undefined unless otherwise specified.
\begin{note}
This applies even to objects such as mutexes intended for thread synchronization.
\end{note}

\rSec3[res.on.requirements]{Semantic requirements}

\pnum
A sequence \tcode{Args} of template arguments is said to
\indextext{concept!model}%
\defnx{model}{model!concept} a concept \tcode{C}
if \tcode{Args}
satisfies \tcode{C}\iref{temp.constr.decl} and
meets all semantic requirements (if any)
given in the specification of \tcode{C}.

\pnum
If the validity or meaning of a program
depends on whether a sequence of template arguments models a concept, and
the concept is satisfied but not modeled,
the program is ill-formed, no diagnostic required.

\pnum
If the semantic requirements of a declaration's
constraints\iref{structure.requirements} are not modeled at the point of use,
the program is ill-formed, no diagnostic required.

\rSec2[conforming]{Conforming implementations}

\rSec3[conforming.overview]{Overview}

\pnum
Subclause \ref{conforming} describes the constraints upon, and latitude of, implementations of the \Cpp{} standard library.

\pnum
An implementation's use of headers is discussed in~\ref{res.on.headers}, its use
of macros in~\ref{res.on.macro.definitions}, non-member functions
in~\ref{global.functions}, member functions in~\ref{member.functions}, data race
avoidance in~\ref{res.on.data.races}, access specifiers
in~\ref{protection.within.classes}, class derivation in~\ref{derivation}, and
exceptions in~\ref{res.on.exception.handling}.

\rSec3[res.on.headers]{Headers}

\pnum
A \Cpp{} header may include other \Cpp{} headers.
A \Cpp{} header shall provide the declarations and definitions that appear in its
synopsis. A \Cpp{} header shown in its synopsis as including other \Cpp{} headers
shall provide the declarations and definitions that appear in the synopses of
those other headers.

\pnum
Certain types and macros are defined in more than one header.
Every such entity shall be defined such that any header that defines it may be
included after any other header that also defines it\iref{basic.def.odr}.

\pnum
The C standard library headers\iref{depr.c.headers}
shall include only their corresponding \Cpp{} standard library header,
as described in~\ref{headers}.

\rSec3[res.on.macro.definitions]{Restrictions on macro definitions}
\indextext{restriction}%

\pnum
The names and global function signatures described in~\ref{contents} are
reserved to the implementation.
\indextext{argument}%
\indextext{header!C}%
\indextext{function!global}%
\indextext{inline}%
\indextext{macro!masking}%

\pnum
All object-like macros defined by the C standard library and described in this
Clause as expanding to integral constant expressions are also suitable for use
in \tcode{\#if}\indextext{\idxcode{\#if}} preprocessing directives, unless
explicitly stated otherwise.

\rSec3[global.functions]{Non-member functions}

\pnum
It is unspecified whether any
non-member
functions in the \Cpp{} standard library are defined as
inline\iref{dcl.inline}.

\pnum
A call to a non-member function signature
described in \ref{\firstlibchapter} through \ref{\lastlibchapter} and
\ref{depr} shall behave as if the implementation declared no additional
non-member function signatures.\footnote{A valid \Cpp{} program always
calls the expected library non-member function. An implementation can
also define additional non-member functions that would otherwise not
be called by a valid \Cpp{} program.}

\pnum
An implementation shall not declare a non-member function signature
with additional default arguments.

\pnum
Unless otherwise specified,
calls made by functions in the standard library to non-operator, non-member functions
do not use functions from another namespace which are found through
argument-dependent name lookup\iref{basic.lookup.argdep}.
\begin{note}
The phrase ``unless otherwise specified'' applies to cases such as
the swappable with requirements\iref{swappable.requirements}.
The exception for overloaded operators allows argument-dependent lookup
in cases like that of
\tcode{ostream_iterator::operator=}\iref{ostream.iterator.ops}:

\effects
\begin{codeblock}
*@\textit{out_stream}@ << value;
if (@\textit{delim}@ != 0)
  *@\textit{out_stream}@ << @\textit{delim}@;
return *this;
\end{codeblock}
\end{note}

\rSec3[member.functions]{Member functions}

\pnum
It is unspecified whether any member functions in the \Cpp{} standard library are defined as
inline\iref{dcl.inline}.

\pnum
For a non-virtual member function described in the \Cpp{} standard library,
an implementation may declare a different set of member function signatures,
provided that any call to the member function that would select
an overload from the set of declarations described in this document
behaves as if that overload were selected.
\begin{note}
For instance, an implementation can add parameters with default values,
or replace a member function with default arguments
with two or more member functions with equivalent behavior,
or add additional signatures for a member function name.
\end{note}

\rSec3[hidden.friends]{Friend functions}

\pnum
Whenever this document specifies
a friend declaration of a function or function template
within a class or class template definition,
that declaration shall be
the only declaration of that function or function template
provided by an implementation.
\begin{note}
In particular,
an implementation is not allowed to provide
an additional declaration of that function or function template
at namespace scope.
\end{note}
\begin{note}
Such a friend function or function template declaration
is known as a hidden friend,
as it is visible neither
to ordinary unqualified lookup\iref{basic.lookup.unqual} nor
to qualified lookup\iref{basic.lookup.qual}.
\end{note}

\rSec3[constexpr.functions]{Constexpr functions and constructors}

\pnum
This document explicitly requires that certain standard library functions are
\tcode{constexpr}\iref{dcl.constexpr}. An implementation shall not declare
any standard library function signature as \tcode{constexpr} except for those where
it is explicitly required.
Within any header that provides any non-defining declarations of constexpr
functions or constructors an implementation shall provide corresponding definitions.

\rSec3[algorithm.stable]{Requirements for stable algorithms}

\pnum
\indextext{algorithm!stable}%
\indextext{stable algorithm}%
When the requirements for an algorithm state that it is ``stable'' without further elaboration,
it means:

\begin{itemize}
\item For the sort algorithms the relative order of equivalent
elements is preserved.

\item For the remove and copy algorithms the relative order of
the elements that are not removed is preserved.

\item For the merge algorithms, for equivalent elements in
the original two ranges, the elements from the first range (preserving their
original order) precede the elements from the second range (preserving their
original order).
\end{itemize}

\rSec3[reentrancy]{Reentrancy}

\pnum
Except where explicitly specified in this document, it is \impldef{which functions in
the \Cpp{} standard library may be recursively reentered} which functions in the \Cpp{} standard
library may be recursively reentered.

\rSec3[res.on.data.races]{Data race avoidance}

\pnum
This subclause specifies requirements that implementations shall meet to prevent data
races\iref{intro.multithread}.
Every standard library function shall meet each requirement unless otherwise specified.
Implementations may prevent data races in cases other than those specified below.

\pnum
A \Cpp{} standard library function shall not directly or indirectly access
objects\iref{intro.multithread} accessible by threads other than the current thread
unless the objects are accessed directly or indirectly via the function's arguments,
including \tcode{this}.

\pnum
A \Cpp{} standard library function shall not directly or indirectly modify
objects\iref{intro.multithread} accessible by threads other than the current thread
unless the objects are accessed directly or indirectly via the function's non-const
arguments, including \tcode{this}.

\pnum
\begin{note}
This means, for example, that implementations can't use an object with static storage duration for
internal purposes without synchronization because it could cause a data race even in
programs that do not explicitly share objects between threads.
\end{note}

\pnum
A \Cpp{} standard library function shall not access objects indirectly accessible via its
arguments or via elements of its container arguments except by invoking functions
required by its specification on those container elements.

\pnum
Operations on iterators obtained by calling a standard library container or string
member function may access the underlying container, but shall not modify it.
\begin{note}
In particular, container operations that invalidate iterators conflict
with operations on iterators associated with that container.
\end{note}

\pnum
Implementations may share their own internal objects between threads if the objects are
not visible to users and are protected against data races.

\pnum
Unless otherwise specified, \Cpp{} standard library functions shall perform all operations
solely within the current thread if those operations have effects that are
visible\iref{intro.multithread} to users.

\pnum
\begin{note}
This allows implementations to parallelize operations if there are no visible
\indextext{side effects}%
side effects.
\end{note}

\rSec3[protection.within.classes]{Protection within classes}

\pnum
\indextext{protection}%
It is unspecified whether any function signature or class described in
\ref{\firstlibchapter} through \ref{\lastlibchapter} and \ref{depr} is a
friend of another class in the \Cpp{} standard library.
\indextext{specifier!\idxcode{friend}}

\rSec3[derivation]{Derived classes}

\pnum
\indextext{class!derived}%
\indextext{class!base}%
An implementation may derive any class in the \Cpp{} standard library from a class with a
name reserved to the implementation.

\pnum
Certain classes defined in the \Cpp{} standard library are required to be derived from
other classes
in the \Cpp{} standard library.
\indextext{library!\Cpp{} standard}%
An implementation may derive such a class directly from the required base or indirectly
through a hierarchy of base classes with names reserved to the implementation.

\pnum
In any case:
\begin{itemize}
\item
Every base class described as
\tcode{virtual}
shall be virtual;
\indextext{class!base}%
\item
Every base class not specified as
\tcode{virtual} shall not be virtual;
\item
Unless explicitly stated otherwise, types with distinct names shall be distinct
types.\footnote{There is an implicit exception to this rule for types that are
described as synonyms for basic integral types, such as
\tcode{size_t}\iref{support.types} and
\tcode{streamoff}\iref{stream.types}.}
\end{itemize}

\pnum
All types specified in the \Cpp{} standard library shall be non-\tcode{final} types
unless otherwise specified.

\rSec3[res.on.exception.handling]{Restrictions on exception handling}%
\indextext{restriction}%
\indextext{exception handling!handler}

\pnum
Any of the functions defined in the \Cpp{} standard library
\indextext{library!\Cpp{} standard}%
can report a failure by throwing an exception of a type
described in its \throws paragraph,
or of a type derived from a type named in the \throws paragraph
that would be caught by an exception handler for the base type.

\pnum
Functions from the C standard library shall not throw exceptions%
\indextext{specifications!C standard library exception}\footnote{That is, the C
library functions can all be treated as if they
are marked \tcode{noexcept}.
This allows implementations to make performance optimizations
based on the absence of exceptions at runtime.}
except when such a function calls a program-supplied function that throws an
exception.\footnote{The functions
\tcode{qsort()}
and
\tcode{bsearch()}\iref{alg.c.library} meet this condition.}

\pnum
Destructor operations defined in the \Cpp{} standard library
shall not throw exceptions.
Every destructor in the \Cpp{} standard library shall behave as if it had a
non-throwing exception specification.

\pnum
Functions defined in the
\Cpp{} standard library
\indextext{specifications!\Cpp{}}%
that do not have a \throws paragraph
but do have a potentially-throwing exception specification
may throw \impldef{exceptions thrown by standard library functions that have a
potentially-throwing exception specification} exceptions.%
\footnote{In particular, they
can report a failure to allocate storage by throwing an exception of type
\tcode{bad_alloc},
or a class derived from
\tcode{bad_alloc}\iref{bad.alloc}.}
Implementations should
report errors by throwing exceptions of or derived
from the standard exception classes~(\ref{bad.alloc},
\ref{support.exception}, \ref{std.exceptions}).

\pnum
An implementation may strengthen the
exception specification
for a non-virtual function
by adding a non-throwing exception specification.

\rSec3[res.on.pointer.storage]{Restrictions on storage of pointers}

\pnum
\indextext{traceable pointer object}%
\indextext{pointer!to traceable object}%
Objects constructed by the standard library that may hold a user-supplied pointer value
or an integer of type \tcode{std::intptr_t} shall store such values in a traceable
pointer location\iref{basic.stc.dynamic.safety}.

\rSec3[value.error.codes]{Value of error codes}

\pnum
Certain functions in the \Cpp{} standard library report errors via a
\tcode{std::error_code}\iref{syserr.errcode.overview} object. That object's
\tcode{category()} member shall return \tcode{std::system_category()} for
errors originating from the operating system, or a reference to an
\impldef{\tcode{error_category} for errors originating outside the
operating system} \tcode{error_category} object for errors originating elsewhere.
The implementation shall define the possible values of \tcode{value()} for each of these
error categories.
\begin{example}
For operating systems that are based on POSIX,
implementations should define the \tcode{std::system_category()} values as
identical to the POSIX \tcode{errno} values, with additional values as defined by the
operating system's documentation. Implementations for operating systems that are not
based on POSIX should define values identical to the operating system's
values. For errors that do not originate from the operating system, the implementation
may provide enums for the associated values.
\end{example}

\rSec3[lib.types.movedfrom]{Moved-from state of library types}

\pnum
Objects of types defined in the \Cpp{} standard library may be moved
from\iref{class.copy.ctor}. Move operations may be explicitly specified or
implicitly generated. Unless otherwise specified, such moved-from objects shall
be placed in a valid but unspecified state.
