%!TEX root = std.tex
\chapter{Foreword}

ISO (the International Organization for Standardization)
is a worldwide federation of national standards bodies (ISO member bodies).
The work of preparing International Standards
is normally carried out through ISO technical committees.
Each member body interested in a subject
for which a technical committee has been established
has the right to be represented on that committee.
International organizations,
governmental and non-governmental,
in liaison with ISO,
also take part in the work.
ISO collaborates closely with
the International Electrotechnical Commission (IEC)
on all matters of electrotechnical standardization.

%ISO (the International Organization for Standardization) and IEC (the
%International Electrotechnical Commission) form the specialized system for
%worldwide standardization. National bodies that are members of ISO or IEC
%participate in the development of International Standards through technical
%committees established by the respective organization to deal with particular
%fields of technical activity. ISO and IEC technical committees collaborate in
%fields of mutual interest. Other international organizations, governmental and
%non-governmental, in liaison with ISO and IEC, also take part in the work. In
%the field of information technology, ISO and IEC have established a joint
%technical committee, ISO/IEC JTC 1.

The procedures used to develop this document and those intended for its further
maintenance are described in the ISO/IEC Directives, Part 1. In particular the
different approval criteria needed for the different types of ISO documents should
be noted. This document was drafted in accordance with the editorial rules of
the ISO/IEC Directives, Part 2
(see \href{http://www.iso.org/directives}{\tcode{www.iso.org/directives}}).

Attention is drawn to the possibility that some of the elements of this
document may be the subject of patent rights. ISO shall not be held
responsible for identifying any or all such patent rights. Details of any
patent rights identified during the development of the document will be in the
Introduction and/or on the ISO list of patent declarations received
(see \href{http://www.iso.org/patents}{\tcode{www.iso.org/patents}}).

Any trade name used in this document is information given for the convenience
of users and does not constitute an endorsement.

For an explanation on
the voluntary nature of standards,
the meaning of ISO specific terms and expressions related
to conformity assessment, as well as information about ISO's adherence
to the World Trade Organization (WTO) principles
in the Technical Barriers to Trade (TBT) see the following URL:
\href{http://www.iso.org/iso/foreword.html}{\tcode{www.iso.org/iso/foreword.html}}.

This document was prepared by
Technical Committee ISO/IEC JTC 1, \textit{Information technology},
Subcommittee SC 22, \textit{Programming languages, their environments and system software interfaces}.

This fifth edition cancels and replaces the fourth edition
(ISO/IEC 14882:2014), which has been technically revised.

The main changes compared to the previous edition are as follows:

\begin{itemize}
% language changes
\item expression evaluation order is specified in more cases
\item removal of trigraphs
\item adjustments to value categories resulting in copy elision being mandatory
\item additional character and floating point literal syntaxes
\item lambda expressions extended to permit capture of \tcode{*this} and use in constant expressions
\item initializer statements for \tcode{if} and \tcode{switch} statements
\item addition of constexpr if statements
\item range-based \tcode{for} statement generalized to support heterogeneous \tcode{begin} and \tcode{end} types
\item addition of structured bindings
\item addition of inline variables
\item list initialization extended to support enumerations and aggregates with base classes
\item message in \tcode{static_assert} is now optional
\item addition of nested namespace definition syntax
\item extended support for attributes
\item exception specifications are now part of function types
\item template argument deduction is now supported for class templates
\item addition of fold expressions
\item pack expansion can be performed on using declarations
\item permitted forms of template parameters and template arguments have been generalized
\item dynamic allocation is supported for over-aligned types
\item preprocessor can detect presence of header files with \tcode{__has_include}
% library changes
\item new utility functions, types, and templates in the standard library, including
\begin{itemize}
  \item an \tcode{any} type
  \item an \tcode{optional} class template
  \item a \tcode{variant} class template
  \item a \tcode{clamp} function
  \item a \tcode{std::byte} type
  \item a \tcode{not_fn} function
  \item a \tcode{void_t} alias template
  \item \tcode{conjunction}, \tcode{disjunction}, and \tcode{negation} templates
  \item an \tcode{invoke} function, and \tcode{is_invocable} and \tcode{invoke_result} type traits
  \item an \tcode{is_swappable} type trait
\end{itemize}
\item extended constant expression evaluation support in the standard library
\item elementary conversion functions between strings and numeric types added
\item constructors for \tcode{pair} and \tcode{tuple} are conditionally-explicit
\item \tcode{shared_ptr}{s} of array types now supported
\item additional algorithms for managing uninitialized memory
\item addition of polymorphic memory resources
\item addition of substring search facilities providing the Boyer-Moore and Boyer-Moore-Horspool search algorithms
\item addition of variable templates for type traits
\item addition of a non-owning string view template
\item ability to splice elements between containers for maps and sets
\item better support for element insertion in unique-key maps
\item support for incomplete types in containers
\item addition of parallel algorithms
\item addition of \tcode{sample} algorithm
\item addition of mathematical special functions, and \tcode{gcd}, \tcode{lcm}, and three-argument \tcode{hypot} functions
\item addition of support for operations on file systems
\item addition of shared mutexes and variadic lock guards
% annexes
\item removal of deprecated features
\end{itemize}
